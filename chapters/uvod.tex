\chapter{Úvod}

\emph{Optika} a \emph{optický systém} jsou pojmy, pod kterými si nejčastěji vybavíme objektiv fotoaparátu, dalekohled nebo brýle. Ovšem realita tohoto odvětví fyziky je poněkud složitější, celé se totiž rozděluje několik různých částí.

Velká část středoškolských znalostí, co se světelných soustav týče, se řadí do kategorie \emph{zobrazovací optiky}. Spadají do ní i všechny výše uvedené předměty a obecně se specializuje na vytváření obrazu nějaké předlohy.

Jistý protipól tvoří \emph{nezobrazovací optika}, ta se zabývá spíše distribucí světla od zdroje určitým zadaným způsobem. Její systémy jsou podstatně rozšířenější, ale zato zpravidla mnohem méně nápadné. Počítáme mezi ně nejrůznější svítilny, světlomety aut, reflektory v divadlech, osvětlení budov, ale i optická vlákna a solární panely nebo např. displeje. (Zde je dobré si uvědomit, že na displeji sice vidíme jakýsi obraz. Ten je ovšem tvořen pouze rozprostřením barevných bodů, pixelů. Optické prvky sice se používají k podsvícení a k distribuci paprsků z jednotlivých pixelů do prostoru, ale ne přímo pro vytvoření obrazu.)

Historicky se optické systémy navrhovaly zejména metodou \emph{pokus-omyl}, dokud se dosahované výsledky nepřiblížily záměru (srov. první dalekohledy a mikroskopy na začátku 17. století). V této době neexistovaly v podstatě téměř žádné pomůcky ani vzorce, na základě kterých by se daly potřebné tvary čoček a zrcadel vypočítat. 

Pro \emph{zobrazovací optiku} nastala změna v průběhu 19. století. Díky novým objevům bylo umožněno matematicky modelovat různé optické soustavy a předejít tak slepému zkoušení různých konfigurací. \parencite{hannavy2008encyclopedia,wimmer2017carl}

Naopak \emph{nezobrazovací optika} žádnou podobnou revoluci neprodělala a i dnes bývá proces navrhování takových systémů z větší části založen na náhodném zkoušení možností. To je způsobeno zejména její značnou složitostí. Jako nečekaně komplexní se prokazuje i tato základní úloha:

\begin{quote}
    Navrhněte reflektor o maximálních rozměrech $20\times20\times20\ \mathrm{mm^3}$ pro LED světelný zdroj $3\times3\ \mathrm{mm^2}$, který vytvoří kužel světla o šířce $10^{\circ}$.
\end{quote}

Nejenže neexistuje žádný přímý způsob, jak se jednoduše dobrat výsledku. Se současnými metodami ani nedokážeme rozhodnout, jestli je úloha vůbec řešitelná. (P. Lobaz, osobní komunikace, 11. 2. 2023)

Obecně je při návrhu optické soustavy potřeba nejprve problém zjednodušit a vymyslet základní koncept řešení, které se posléze dále vylepšuje. Při práci se \emph{zobrazovacím systémem} často používáme idealizované tenké čočky, dalším způsobem může být výpočet pouze na základě několika významných paprsků apod.

Obdobných pomůcek z oblasti \emph{nezobrazovací optiky} není mnoho. A pokud existují, zpravidla nejsou dostupné v žádném komerčním softwaru pro návrh optických soustav (např. \emph{TracePro}, \emph{LightTools} nebo \emph{Photopia}). (P. Lobaz, osobní komunikace, 11. 2. 2023)

Jedním z nástrojů, který by mohl pomoci zjednodušit a urychlit proces návrhu nezobrazovacích systémů, je i vizualizace tzv. \emph{prostorovo-fázového diagramu} (anglicky \emph{phase-space diagram}). \parencite{mushaveck2022designing} Cílem této práce je zjistit, zda-li počítačový program, který umožňuje interaktivně zobrazit právě takovýto diagram, dokáže přispět k lepšímu pochopení navrhované soustavy a případně k jejímu vylepšení.