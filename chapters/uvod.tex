\chapter{Úvod}

\fixme{Vlastními slovy}

Pod pojmem \emph{optika} a \emph{optická soustava} si nejčastěji představujeme brýle, dalekohledy, fotoaparáty, mikroskopy apod. To všechno jsou příklady tzv. \emph{zobrazovací optiky} - taková optická soustava má vytvořit obraz předlohy. Mnohem nenápadnější, ale podstatně rozšířenější je tzv. \emph{nezobrazovací optika} - svítilny, automobilové světlomety, osvětlení budov, světlovody a optická vlákna, světelné zdroje, displeje, solární panely atd.

Návrh zobrazovací optiky byl zpočátku (17. století - dalekohled, mikroskop) v podstatě založen na metodě pokus-omyl. Až objevy v 19. století umožnily založit návrh zobrazovací optiky na matematických modelech a slepé zkoušení všemožných kombinací čoček, clon a zrcadel skončilo. Návrh zobrazovací optiky se stal sice složitou, ale relativně rutinní prací s dobře podchycenou metodikou.

Naproti tomu návrh nezobrazovací optiky je překvapivě z velké části stále ve fázi \emph{metoda pokus-omyl}, zejména pro její značnou složitost. Nečekaně komplexní je například i základní úloha nezobrazovací optiky:

\begin{quote}
    Vytvořte reflektor o maximálních rozměrech $20\times20\times20\ mm^3$ pro LED světelný zdroj $3\times3\ mm^2$, který vytvoří kužel světla o šířce $10^{\circ}$.
\end{quote}

Nejenže neexistuje metodika, jak dospět k cíli, dokonce ani neexistují postupy, jak zjistit, zda je úloha s takovým zadáním vůbec řešitelná.

Návrhář optického systému potřebuje v nejprve problém zjednodušit a vymyslet jeho koncepci. V případě zobrazovací optiky se nejčastěji používá náhrada reálných čoček jejich tenkými idealizacemi, sledování chodu několika významných paprsků, výpočet poloh hlavních bodů apod.

Pro návrhy nezobrazovacích optických systémů je takových nástrojů k dispozici jen velmi málo a v podstatě žádná z nich není součástí žádného komerčního softwaru pro návrh nezobrazovací optiky (\emph{TracePro}, \emph{LightTools}, \emph{Photopia} apod.).

Jednou ze těchto pomůcek by mohla být vizualizace tzv. \emph{prostorovo-fázového diagramu} (\emph{phase-space diagram}). Cílem této práce je zjistit, zda interaktivní zobrazení tohoto diagramu přispěje k lepšímu pochopení detailů funkce navrhovaného optického systému a případně k jeho vylepšení.
