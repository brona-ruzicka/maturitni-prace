\chapter{Úvod}

Pod pojmem "optika" a "optická soustava" si nejčastěji představujeme
brýle, dalekohledy, fotoaparáty, mikroskopy apod. To všechno jsou
příklady tzv. zobrazovací optiky, tj. optická soustava má vytvořit obraz
předlohy. Mnohem nenápadnější, ale podstatně rozšířenější je tzv.
nezobrazovací optika -- svítilny, automobilové světlomety, osvětlení
budov, světlovody a optická vlákna, světelné zdroje, displeje, solární
panely atd.

Návrh zobrazovací optiky byl zpočátku (17. století -- dalekohled,
mikroskop) v podstatě založen na metodě pokus-omyl. Až objevy v 19.
století umožnily založit návrh zobrazovací optiky na matematických
modelech a slepé zkoušení všemožných kombinací čoček, clon a zrcadel
skončilo. Návrh zobrazovací optiky se stal sice složitou, ale relativně
rutinní prací s dobře podchycenou metodikou.

Naproti tomu návrh nezobrazovací optiky je překvapivě z velké části
stále ve fázi "metoda pokus-omyl", zejména pro její značnou složitost.
Nečekaně složitá je například základní úloha nezobrazovací optiky
"vytvořte reflektor o maximálních rozměrech 20x20x20 mm3 pro LED
světelný zdroj 3x3 mm2, který vytvoří kužel světla o šířce 10°". Nejenom
že neexistuje metodika, jak dospět k cíli -- dokonce ani neexistují
postupy, jak zjistit, zda je úloha s takovým zadáním vůbec řešitelná.

Návrhář optického systému potřebuje v první fázi problém zjednodušit a
vymyslet jeho koncepci. V případě zobrazovací optiky se nejčastěji
používá náhrada reálných čoček jejich tenkými idealizacemi, sledování
chodu několika významných paprsků, výpočet poloh hlavních bodů a pupil
apod. Návrhář nezobrazovacích optických systémů má takových pomůcek k
dispozici málo a v podstatě žádná z nich není součástí žádného
komerčního softwaru pro návrh nezobrazovací optiky (TracePro,
LightTools, Photopia apod.).

Jednou z takových pomůcek by mohla být vizualizace prostorovo-fázového
diagramu (phase-space diagram). Cílem této práce je zjistit, zda
interaktivní (bude interaktivní?) vizualizace prostorovo-fázového
diagramu vede k lepšímu pochopení detailů funkce navrhovaného optického
systému a případně k jeho vylepšení.
