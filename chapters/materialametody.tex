\chapter{Materiál a metody}

\section{Vymezení problému}

Smyslem této práce je pouze demonstrovat jistý koncept. Zaměříme proto jen na úplné jádro problému, a pokud se výsledek prokáže jako užitečný, můžeme uvažovat o nějakém zobecnění. Připomeňme si, že cílem je jednak samozřejmě pochopení celkové funkce navrhovaného systému, ovšem neméně nás zajímají i detaily o vystupujícím světle.

Příkladem, na kterém si ukážeme postupné zjednodušení, budiž tento jednoduchý světlovod. Bude mít tvar \emph{pláště komolého jehlanu (případně hranolu) s odříznutými podstavami}. Namísto horní podstavy umístíme \emph{plošný zdroj světla}. Vysílané paprsky se z něj šíří skrz kužel, a to jednak přímo, ale i s pomocí odrazů od stěn.

V prvním kroku se omezíme pouze na podélný řez této soustavy, jinými slovy budeme se pohybovat pouze ve 2D. Zdrojem světla bude nyní jen úsečka a celý světlovod se zjednoduší na dvě úsečky reprezentující zrcadla:

\image[scale=0.7]{vymezeniproblemu_priklad.png}{Jednoduchý 2D světlovod tvořený dvěma zrcadly (modrá) s úsečkovým zdrojem světla a několika znázorněnými paprsky (červená)}

V návaznosti je nutné podotknout, že výsledky simulace ve 2D se mohou v určitých směrech zásadně lišit od těch pro 3D. Jako jeden z rozdílů můžeme uvést fyzikální veličinu osvětlenost ( $ [\ E\ ] = 1 \ \mathrm{\frac{lm}{m^2}} $ ). Ve třech prostorových dimenzích se snižuje s druhou mocninou vzdálenosti (podle \emph{zákona převrácených čtverců}, anglicky \emph{inverse-square law}), zatímco její dvourozměrný protějšek se bude zmenšovat pouze lineárně.

Zároveň zdůrazněme, že nebudeme nahrazovat \emph{plošný zdroj} světla \emph{bodovým}. Tato metoda je velmi typická pro zobrazovací optiku, ale v oblasti nezobrazovací optiky se příliš nepoužívá. Hlavním důvodem je velikost světelných zdrojů, která bývá v navrhovaných systémech zpravidla nezanedbatelná vůči velikosti celé soustavy.

Dále se omezíme na studium makroskopických soustav (svítilna, světlomet). V takto velkých systémech se v podstatě neprojevují (resp. projevují, ale zanedbatelně) \emph{vlnové vlastnosti světla}, které proto ze simulace můžeme vypustit. Ve výsledku tedy budeme na světlo nahlížet pouze z pohledu \emph{paprskové optiky}. To by nebylo možné např. při zkoumání optických vláken, které se svými rozměry více přibližují vlnové délce použitého světla.

Budeme předpokládat, že všechny paprsky mají stejný, \emph{elementární}, optický výkon. Optický výkon celého světelného zdroje je pak jasně dán pouhým počtem paprsků, které vyzařuje.

Pro jednoduchost budeme v demonstračním programu pracovat pouze se dvěma typy interakcí mezi systémem a paprskem, \emph{dokonalým zrcadlovým odrazem} a \emph{dokonalou absorpci}. Pro praktické využití by se hodilo zahrnout i \emph{lom světla}, \emph{rozptyl} (odraz od matného povrchu), \emph{částečný odraz} aj. Zanedbáme též závislost zmíněných jevů na polarizaci nebo vlnové délce (barvě) světla.

\image[scale=0.7]{vymezeniproblemu_reflektor.png}{Příklad dokonalého odrazu na zrcadle (modrá) s bodovým zdrojem světla a několika znázorněnými paprsky (červená)}

\image[scale=0.7]{vymezeniproblemu_absorber.png}{Příklad dokonalé absorbce na cloně (šedá) s bodovým zdrojem světla a několika znázorněnými paprsky (červená)}

Reálné optické prvky bývají ve většině případů tvarované podle nejrůznějších matematických předpisů (např. často $y = k \cdot x^2$). Ovšem jelikož výpočty s mnoha typy křivek a funkcí nebývají vždy snadné, budeme připouštět pouze aproximaci těchto tvarů pomocí lomených čar (série mnoha krátkých úseček). Při budoucím použití bylo vhodné podporovat alespoň některé hojně používané útvary (\emph{kvadratická funkce}, \emph{bezierovy křivky} aj.).

\image[scale=0.7]{vymezeniproblemu_aproximace.png}{Aproximace zrcadla ve tvaru kvadratické funkce (modrá) s úsečkovým zdrojem světla a několika znázorněnými paprsky (červená)}

Podstatný rozdíl je, že v nezobrazovací systémech zpravidla není zřejmé, jak přesně bude který paprsek postupovat optickou soustavou. Oproti tomu v rámci zobrazovací optiky je to velmi jednoduše vyvoditelné. Jako příklad si vezměme teleskop se dvěma čočkami. U něj je na první pohled jasné, že paprsek světla nejdříve projde přes přední, a pak zadní plochu první čočky, a poté to samé pro čočku druhou. Zejména díky této znalosti je analýza i návrh zobrazovacích soustav podstatně jednodušší, než je tomu u systémů nezobrazovacích.






\section{Analýza systému}

\fixme{Vlastními slovy}

Budeme používat dvourozměrný kartézský souřadný systém (systém $Oxy$), 
přičemž počátek soustavy $O$ můžeme umístit kamkoliv. Předpokládáme, že světlo
se primárně šíří ve směru $+x$, světelný paprsek tak můžeme
popsat rovnicí: 

$$y = kx + q$$

K analýze paprsků v místech, kde protínají osu $y$ systému $Oxy$,
slouží \emph{prostorovo-fázový diagram} (\emph{pf-diagram}).


\todo{Vysvětlení a obrázky: jeden paprsek,
vějíř paprsků vycházejících z jednoho bodu, svazek rovnoběžných paprsků,
plošný zdroj světla. Vysvětlení může být na diagramu úhel-místo,
případně směrnice-místo; uvést, že nejpraktičtější je diagram "sinus
úhlu"-místo, resp. "y složka směrového vektoru paprsku"-místo.}

\todo{Možno vysvětlit, že v případě 3D geometrie je prostorovo-fázový diagram
čtyřrozměrný (dvě prostorové, dvě úhlové veličiny), což významně
komplikuje vizualizaci. I proto omezení na 2D; pokud bude 2D analýza
neužitečná, tím spíš bude neužitečná analýza 3D.}

\todo{Ukázat, co se děje, pokud počátek systému $Oxy$ posuneme
doprava nebo doleva -- vodorovné nebo svislé "zkosení" obrazce v
prostorovo-fázovém diagramu. Pozn.: opravdové zkosení je to v případě
diagramu směrnice-místo; v případě "sinus úhlu"-místo se do zkosení
přidává i nelineární zkřivení.}

Ukázat, co se děje při odrazu na zrcadle, a to jak v případě, že je
odražen kompletní svazek, tak v případě, že je zrcadlo malé a je
odražena pouze část svazku.

Dokud platí, že při interakci s optickým povrchem jeden vstupní paprsek
vytváří jeden výstupní paprsek, je plocha obrazce tvořeného všemi
paprsky v prostorovo-fázovém diagramu konstantní nezávisle na volbě
počátku systému $Oxy$. To je zákon zachování étendue; étendue
je plocha obrazce v prostorovo-fázovém diagramu.

Zjednodušeně lze říct, že součin prostorového a úhlového rozsahu je
konstantní. Přesněji řečeno to platí jen pro svazek paprsků
diferenciální velikosti (nekonečně malý rozsah v úhlové i prostorové
souřadnici), ale jelikož celé étendue je složeno z nekonečně mnoha
svazků diferenciální velikosti, platí to integrálně (v sumě) i pro celý
svazek paprsků.

\pic{Možno dovysvětlit obrázkem -- těsně u světelného zdroje
je prostorový rozsah světelného svazku malý a úhlový velký; daleko od
světelného zdroje se světlo hodně roztáhne, tj. prostorový rozsah je
velký, ale lokálně je úhlový rozsah malý.}

Z toho se dá například pro náš jednoduchý světlovod odhadnout, že čím
bude výstupní otvor větší, tím bude mít vystupující světlo menší úhlový
rozsah -- tady by bylo dobré dát obrázky.

Pokud ale chceme za jednoduchý světlovod umístit další optický člen, je
užitečné znát nejenom globální chování veškerého světla vycházejícího ze
světlovodu, ale i lokální detaily chování -- například jaké světlo
vystupuje ze středu výstupního otvoru, poblíž okrajů apod. To by měl
zodpovědět prostorovo-fázový diagram.

Kromě vizuálního hodnocení prostorovo-fázového diagramu je při analýze
optického systému vhodné znát i další údaje, a to:
-- étendue
-- celkový světelný tok procházející osou y v systému Oxy
-- rozložení svítivosti, tj. kolik světelného toku jde do kterého směru

Poznámka k jednotkám:
- Optický výkon, přesněji zářivý výkon, se udává ve wattech [$W$]. V
případě světelnětechnických výpočtů, kde se bere v potaz i citlivost
lidského oka na světlo různých vlnových délek, se přechází z wattů [$W$]
na lumeny [$lm$]. V případě aproximace světla konečným množstvím paprsků
můžeme říkat, že paprsek nese elementární optický výkon ve wattech resp.
lumenech.
- Jelikož je étendue součinem prostorového a úhlového rozsahu, často se
za jednotku étendue bere metr čtverečný krát steradián [$m^2 \cdot sr$]. Při
přechodu na 2-D optický systém bude jednotkou metr krát radián [$m \cdot rad$].
Pozorný čtenář by mohl namítnout, že prostorovo-fázový diagram,
kde se étendue počítá, má osu "sinus úhlu" a nikoliv "úhel". To na věci
nic nemění, jen je třeba pečlivě formulovat, jak se má jednotka "radián"
v jednotce étendue chápat.
- Svítivost je úhlová hustota světla uváděná v lumenech na steradián
[$lm/sr$] čili kandelách [$cd$]; v případě radiometrických veličin hovoříme
o zářivosti s jednotkou watt na steradián [$W/sr$]. Po přechodu na 2-D
optický systém bude jednotkou watt na radián [$W/rad$], resp. lumen na
radián [$lm/rad$].

Také je vhodné si uvědomit, že některé poučky formulované pro 3-D
geometrii neplatí ve 2-D. Například ve 3-D platí, že osvětlenost
[$lm/m^2$] klesá se čtvercem vzdálenosti od světelného zdroje. V případě
2-D optických systémů klesá lineárně se vzdáleností od světelného
zdroje. Výsledky analýzy 2-D systémů tedy nejdou bezprostředně použít
pro 3-D systémy.



\section{Implementační prostředky}

\fixme{Vlastními slovy}

Tady bych se moc nerozepisoval. Java + Swing proto, že Javu znáte a na
Swing existuje dobrá dokumentace a je ve standardním SDK. Ad IDE - není
důležité, můžete zmínit; jediný důvod pro výběr asi je, že jej znáte.
Cílem bylo rychle prověřit použitelnost nápadu, nikoliv vytvořit
produkční software.



\section{Architektura aplikace}

\fixme{Vlastními slovy}

Popis by měl být přizpůsobený Vaší implementaci; níže je, jak bych to
implementoval já. Pokud to uznáte za vhodné, můžete jít až na
matematické vztahy, kterými něco počítáte, ale odkaz na literaturu je
dostatečný -- jde o standardní operace, které není nutné rozpitvávat.
Uznáte-li za vhodné přidat obrázky, přidejte.

Sledování paprsku (ray tracing) - ze zdroje světla se vyšle paprsek,
spočítá se průsečík se všemi objekty, vybere se nejbližší, vypočítá se
interakce (odraz, absorpce) a případně se ve sledování paprsku pokračuje
dál, dokud se nedosáhne limitního počtu interakcí (např. 1000), nebo
paprsek neopustí scénu (průsečík neexistuje).

Vizualizace chodu paprsků simulovaným systémem - tady asi jen napsat, že
k pochopení chování optického systému je užitečné vidět nejenom
abstraktní prostorovo-fázový diagram, ale i jednotlivé optické členy a
chod paprsků mezi nimi. A důležitá drobnost: pro výpočet
prostorovo-fázového diagramu potřebujete třeba statisíce paprsků, ale
pro vizualizaci chodu paprsků je vhodné nakreslit jich *maximálně* pár
set, jinak je z toho nepřehledná změť čar. Máte-li ve vizualizaci nějaký
implementačně zajímavý detail, klidně jej samozřejmě uveďte.

Výpočet prostorovo-fázového diagramu - výpočet průsečíku každého z
paprsků (jak polopřímek, tak úseček) s osou y zvoleného souřadného
systému Oxy; pokud existuje, zaznamenat průsečík + směr paprsku do
pomocného pole. Následně určit rozsah prostorovo-fázového diagramu,
rozdělit na pixely, do každého určit, kolik paprsků obsahuje.
Alternativně: rozsah prostorovo-fázového diagramu určit předem
(uživatelsky), aby byl během vizualizace rozsah pořád stejný.

Étendue: počet neprázdných pixelů. Tok: celkový počet paprsků v pomocném
poli průsečíků. Rozložení svítivosti: součet počtu paprsků v každém
diskrétním úhlu (šířky 1 pixel) prostorovo-fázového diagramu. Někde
hlásit, kolik paprsků z pomocného pole průsečíků není v
prostorovo-fázovém diagramu obsaženo.

Nevím, jestli něco obecného psát k obecné architektuře programu - to
musíte posoudit sám, jestli je tam něco mimořádně zajímavého.



% \section{Testování aplikace}
%
% \fixme{Vlastními slovy}
%
% Tato kapitolka jen pokud na ni zbyde dost času. Mohli bychom něco
% odsimulovat třeba v TracePro. Také by šlo vymodelovat jednoduché
% případy, na kterých bude zřejmé, že výsledek je správně.



\section{Ovládání aplikace}

\todo{Hlavně napsat, jak se definuje scéna. Pak případně popsat GUI.}




