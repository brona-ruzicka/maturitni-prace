\chapter{Materiál a metody}

\section{Vymezení problému}

Jako vzorový příklad si můžeme vzít zrcadlový světlovod -- plášť
komolého jehlanu s odříznutou špičkou a podstavou; v limitním případě
může jít o hranol. Ze strany odříznuté špičky svítí plošný zdroj světla.
Světlo se šíří ve směru k odříznuté podstavě, a to jak přímo, tak odrazy
od stěn. Naším cílem je pochopit nejenom celkovou funkci této optické
soustavy, ale i detaily o vystupujícím světle.

Tato práce je pracovním pokusem ověřujícím jistou myšlenku. Proto je
vhodné nejprve problém zásadně zjednodušit a až poté uvažovat o zobecnění.

V této fázi je proto na místě omezit se nejprve na 2-D řez, respektive
na 2-D optický systém. Naše optická soustava se tedy zjednoduší na zdroj
světla v podobě úsečky a světlovod ve formě lichoběžníku (limitně
obdélníku) s odříznutými čely, tj. de facto dvěma úsečkami
představujícími zrcadla:


                     ----
         zrcadlo    /
                ----
               /
          ----
         /

         S
světlo  S
         S

         \
          ----
              \
               ----
         zrcadlo   \
                    ----

(v obrázku by měla být zrcadla skoro rovnoběžná, "lomenou čarou" jsem se
snažil napodobit rovnou čáru malého sklonu)

Nebudeme zavádět zjednodušení typické u zobrazovací optiky -- náhradu
plošného (lineárního) zdroje světla bodovým. Nezobrazovací optika je
typická právě prací se světelnými zdroji, jejichž velikost je
nezanedbatelná vzhledem k velikosti optického systému.

Dále budeme pracovat čistě s paprskovou představou světla, tj. nebudeme
uvažovat jeho vlnové vlastnosti. To je adekvátní u makroskopických
optických soustav (např. světlomety), naopak neadekvátní např. u studia
optických vláken. Takové typy nezobrazovací optiky tedy z analýzy vědomě
vypouštíme.

Budeme uvažovat pouze dokonalý zrcadlový odraz nebo dokonalou absorpci
světla. V praxi by bylo třeba též zahrnout lom světla, rozptyl (odraz od
matného povrchu), částečný odraz, případně další jevy. Nebudeme uvažovat
závislost optické interakce na barvě (vlnové délce) světla nebo jeho
polarizaci. Budeme předpokládat, že každý paprsek nese stejný
elementární optický výkon. Optický výkon (světelný tok) celého
světelného zdroje je pak dán prostým součtem elementárních optických výkonů.

Obrázky - odraz na lineárním zrcadlu, absorpce, případně další.

Sledovaná optická soustava může být velmi jednoduchá jako v uvedeném
případě, ale může také obsahovat neomezené množství lineárních zrcadel
(k aproximaci křivých povrchů), lineárních světelných zdrojů a
lineárních absorbérů (clon). Důležité je, že předem není známo, jak bude
paprsek optickou soustavou postupovat. To je zásadní rozdíl oproti
zobrazovací optice, např. teleskopu se dvěma čočkami, kde je předem
zřejmé, že paprsek světla se nejdřív lomí na přední ploše první čočky,
pak na zadní ploše první čočky, pak na přední ploše druhé čočky a
nakonec na zadní ploše druhé čočky. Tato apriorní znalost podstatně
zjednodušuje analýzu zobrazovací optiky -- a neznalost komplikuje
analýzu nezobrazovací optiky.



\section{Analýza systému}

Budeme předpokládat kartézský souřadný systém Oxy, kde počátek O můžeme
umístit kamkoliv. Budeme předpokládat, že světlo se primárně šíří ve
směru osy +x.

Světelný paprsek tak můžeme v souřadném systému Oxy popsat rovnicí
   y = kx + q.

K analýze paprsků v místech, kde protínají osu y souřadného systému Oxy,
slouží prostorovo-fázový diagram. Vysvětlení a obrázky: jeden paprsek,
vějíř paprsků vycházejících z jednoho bodu, svazek rovnoběžných paprsků,
plošný zdroj světla. Vysvětlení může být na diagramu úhel-místo,
případně směrnice-místo; uvést, že nejpraktičtější je diagram "sinus
úhlu"-místo, resp. "y složka směrového vektoru paprsku"-místo.

Možno vysvětlit, že v případě 3-D geometrie je prostorovo-fázový diagram
čtyřrozměrný (dvě prostorové, dvě úhlové veličiny), což významně
komplikuje vizualizaci. I proto omezení na 2-D: pokud bude 2-D analýza
neužitečná, tím spíš bude neužitečná analýza 3-D.

Ukázat, co se děje, pokud počátek souřadného systému Oxy posuneme
doprava nebo doleva -- vodorovné nebo svislé "zkosení" obrazce v
prostorovo-fázovém diagramu. Pozn.: opravdové zkosení je to v případě
diagramu směrnice-místo; v případě "sinus úhlu"-místo se do zkosení
přidává i nelineární zkřivení.

(Škoda, že jste neimplementoval tenkou čočku - mohl jste ukázat, co se
děje na ní. Ale svět se tím neboří.)

Ukázat, co se děje při odrazu na zrcadle, a to jak v případě, že je
odražen kompletní svazek, tak v případě, že je zrcadlo malé a je
odražena pouze část svazku.

Dokud platí, že při interakci s optickým povrchem jeden vstupní paprsek
vytváří jeden výstupní paprsek, je plocha obrazce tvořeného všemi
paprsky v prostorovo-fázovém diagramu konstantní nezávisle na volbě
počátku souřadného systému Oxy. To je zákon zachování étendue; étendue
je plocha obrazce v prostorovo-fázovém diagramu.

Zjednodušeně lze říct, že součin prostorového a úhlového rozsahu je
konstantní. Přesněji řečeno to platí jen pro svazek paprsků
diferenciální velikosti (nekonečně malý rozsah v úhlové i prostorové
souřadnici), ale jelikož celé étendue je složeno z nekonečně mnoha
svazků diferenciální velikosti, platí to integrálně (v sumě) i pro celý
svazek paprsků. [Možno dovysvětlit obrázkem -- těsně u světelného zdroje
je prostorový rozsah světelného svazku malý a úhlový velký; daleko od
světelného zdroje se světlo hodně roztáhne, tj. prostorový rozsah je
velký, ale lokálně je úhlový rozsah malý.]

Z toho se dá například pro náš jednoduchý světlovod odhadnout, že čím
bude výstupní otvor větší, tím bude mít vystupující světlo menší úhlový
rozsah -- tady by bylo dobré dát obrázky.

Pokud ale chceme za jednoduchý světlovod umístit další optický člen, je
užitečné znát nejenom globální chování veškerého světla vycházejícího ze
světlovodu, ale i lokální detaily chování -- například jaké světlo
vystupuje ze středu výstupního otvoru, poblíž okrajů apod. To by měl
zodpovědět prostorovo-fázový diagram.

Kromě vizuálního hodnocení prostorovo-fázového diagramu je při analýze
optického systému vhodné znát i další údaje, a to:
-- étendue
-- celkový světelný tok procházející osou y v systému Oxy
-- rozložení svítivosti, tj. kolik světelného toku jde do kterého směru

Poznámka k jednotkám:
- Optický výkon, přesněji zářivý výkon, se udává ve wattech [$W$]. V
případě světelnětechnických výpočtů, kde se bere v potaz i citlivost
lidského oka na světlo různých vlnových délek, se přechází z wattů [$W$]
na lumeny [$lm$]. V případě aproximace světla konečným množstvím paprsků
můžeme říkat, že paprsek nese elementární optický výkon ve wattech resp.
lumenech.
- Jelikož je étendue součinem prostorového a úhlového rozsahu, často se
za jednotku étendue bere metr čtverečný krát steradián [$m^2 . sr$]. Při
přechodu na 2-D optický systém bude jednotkou metr krát radián [m . rad].
Pozorný čtenář by mohl namítnout, že prostorovo-fázový diagram,
kde se étendue počítá, má osu "sinus úhlu" a nikoliv "úhel". To na věci
nic nemění, jen je třeba pečlivě formulovat, jak se má jednotka "radián"
v jednotce étendue chápat.
- Svítivost je úhlová hustota světla uváděná v lumenech na steradián
[$lm/sr$] čili kandelách [$cd$]; v případě radiometrických veličin hovoříme
o zářivosti s jednotkou watt na steradián [$W/sr$]. Po přechodu na 2-D
optický systém bude jednotkou watt na radián [$W/rad$], resp. lumen na
radián [$lm/rad$].

Také je vhodné si uvědomit, že některé poučky formulované pro 3-D
geometrii neplatí ve 2-D. Například ve 3-D platí, že osvětlenost
[$lm/m^2$] klesá se čtvercem vzdálenosti od světelného zdroje. V případě
2-D optických systémů klesá lineárně se vzdáleností od světelného
zdroje. Výsledky analýzy 2-D systémů tedy nejdou bezprostředně použít
pro 3-D systémy.



\section{Implementační prostředky}

Tady bych se moc nerozepisoval. Java + Swing proto, že Javu znáte a na
Swing existuje dobrá dokumentace a je ve standardním SDK. Ad IDE - není
důležité, můžete zmínit; jediný důvod pro výběr asi je, že jej znáte.
Cílem bylo rychle prověřit použitelnost nápadu, nikoliv vytvořit
produkční software.



\section{Architektura aplikace}

Popis by měl být přizpůsobený Vaší implementaci; níže je, jak bych to
implementoval já. Pokud to uznáte za vhodné, můžete jít až na
matematické vztahy, kterými něco počítáte, ale odkaz na literaturu je
dostatečný -- jde o standardní operace, které není nutné rozpitvávat.
Uznáte-li za vhodné přidat obrázky, přidejte.

Sledování paprsku (ray tracing) - ze zdroje světla se vyšle paprsek,
spočítá se průsečík se všemi objekty, vybere se nejbližší, vypočítá se
interakce (odraz, absorpce) a případně se ve sledování paprsku pokračuje
dál, dokud se nedosáhne limitního počtu interakcí (např. 1000), nebo
paprsek neopustí scénu (průsečík neexistuje).

Vizualizace chodu paprsků simulovaným systémem - tady asi jen napsat, že
k pochopení chování optického systému je užitečné vidět nejenom
abstraktní prostorovo-fázový diagram, ale i jednotlivé optické členy a
chod paprsků mezi nimi. A důležitá drobnost: pro výpočet
prostorovo-fázového diagramu potřebujete třeba statisíce paprsků, ale
pro vizualizaci chodu paprsků je vhodné nakreslit jich *maximálně* pár
set, jinak je z toho nepřehledná změť čar. Máte-li ve vizualizaci nějaký
implementačně zajímavý detail, klidně jej samozřejmě uveďte.

Výpočet prostorovo-fázového diagramu - výpočet průsečíku každého z
paprsků (jak polopřímek, tak úseček) s osou y zvoleného souřadného
systému Oxy; pokud existuje, zaznamenat průsečík + směr paprsku do
pomocného pole. Následně určit rozsah prostorovo-fázového diagramu,
rozdělit na pixely, do každého určit, kolik paprsků obsahuje.
Alternativně: rozsah prostorovo-fázového diagramu určit předem
(uživatelsky), aby byl během vizualizace rozsah pořád stejný.

Étendue: počet neprázdných pixelů. Tok: celkový počet paprsků v pomocném
poli průsečíků. Rozložení svítivosti: součet počtu paprsků v každém
diskrétním úhlu (šířky 1 pixel) prostorovo-fázového diagramu. Někde
hlásit, kolik paprsků z pomocného pole průsečíků není v
prostorovo-fázovém diagramu obsaženo.

Nevím, jestli něco obecného psát k obecné architektuře programu - to
musíte posoudit sám, jestli je tam něco mimořádně zajímavého.



\section{Testování aplikace}

Tato kapitolka jen pokud na ni zbyde dost času. Mohli bychom něco
odsimulovat třeba v TracePro. Také by šlo vymodelovat jednoduché
případy, na kterých bude zřejmé, že výsledek je správně.



\section{Ovládání aplikace}

Hlavně napsat, jak se definuje scéna. Pak případně popsat GUI.

