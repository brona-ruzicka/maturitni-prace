\chapter{Závěr}

V rámci této maturitní práce jsem se zabýval problematikou návrhu nezobrazovacích optických systémů. Specificky jsem ověřoval užitečnost vizualizace prostorovo-fá\-zo\-vé\-ho diagramu pro 2D systémy.

Vytvořil jsem počítačovou aplikaci, která dokáže sledovat chod paprsků 2D scénou složenou ze zdrojů světla a optických prvků (clona a zrcadlo) vyjádřených úsečkami. Dále jsem implementoval vizualizaci jak samotné scény, tak i průchodu jednotlivých paprsků. Vytvořený program umí samozřejmě zobrazit zmíněný prostorovo-fázový diagram a také vypočítat jeho plochu, zvanou étendue. Poslední implementovanou částí aplikace je funkce vykreslení grafu rozložení svítivosti.

Společně s odborníkem z praxe jsem došel k závěru, že tento typ aplikace se zdá být užitečný. Bylo by proto vhodné se touto problematiku dále zabývat, ať už ve formě dalšího vylepšování simulační aplikace, nebo uskutečnění podobného rozboru užitečnosti i pro 3D systémy.
