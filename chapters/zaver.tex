\chapter{Závěr}

V rámci této maturitní práce jsem se zabýval problematikou návrhu nezobrazovacích optických systémů. Specificky jsem ověřoval užitečnost vizualizace prostorovo-fázového diagramu.

Vytvořil jsem počítačovou aplikaci, která dokázala sledovat chod paprsků 2D scénou složenou z zdrojů světla a optických prvků (clona a zrcadlo) vyjádřených úsečkami. Dále jsem implementoval vizualizaci jak samotné scény tak i chodu jednotlivých paprsků. Vytvořený program uměl samozřejmě zobrazit zmíněný prostorovo-fázový diagram a také vypočítat jeho plochu zvanou étendue. Poslední implementovanou část aplikace byla funkce vykreslení grafu rozložení svítivosti.

Společně s odborníkem z praxe jsem došel k závěru, že tento typ aplikace se zdá být užitečný a že by bylo vhodné zabývat se touto problematikou i dále. Ať už je to ve formě dalšího vylepšování simulační aplikace, nebo uskutečnění podobného rozboru užitečnosti i pro 3D systémy.
