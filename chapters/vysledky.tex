\chapter{Výsledky}

\todo {Popis nasimulovaných obrázků, úprava textu}

Vybrali jsme celkem pět různých optických systémů. Celkově by bylo možné je rozdělit rozdělit do dvou kategorií, \emph{světlovody} a \emph{parabolické reflektory}. Pro každý z těchto systémů jsme spustili náš program a poté uložili vypočtené vizualizace a grafy.


\section{Světlovody}
\label{sec:svetlovody}

\subsection{Krátký světlovod}

První systém je zadán takto:

\begin{quote}
    Jednoduchý zrcadlový světlovod se šířkou vstupu $2\ \mathrm{m}$ a šířkou výstupu $5\ \mathrm{m}$. Jeho celková délka činí $5\ \mathrm{m}$ a na vstupu do světlovodu je připevněn úsečkový zdroj o délce $1,95\ \mathrm{m}$.
\end{quote}

Odpovídající definice scény:

\begin{minipage}{\textwidth}\begin{quote}\begin{lstlisting}
Scene scene = Scene.create(
        // Horní část světlovodu
        Interactors.reflecting(
                Geometries.line(point(0, 1), point(5, 2))
        ),
        // Spodní část světlovodu
        Interactors.reflecting(
                Geometries.line(point(0, -1), point(5, -2))
        ),

        // Zdroj paprsků
        Emitters.line(
                point(0, 0),
                1.95f,
                100000
        )
);
EtendueApp.run(scene);
\end{lstlisting}\end{quote}\end{minipage}

Vygenerované vizualizace:

\singleimage{
    \image[scale=0.15]{vysledky_1a_scena.png}{Schéma krátkého světlovodu}
}

\doubleimage{
    \image[scale=0.15]{vysledky_1a_etendue.png}{Pf-diagram na výstupu z krátkého světlovodu}
}{
    \image[scale=0.15]{vysledky_1a_graf.png}{Rozložení svítivosti po průchodu krátkým světlovodem}
}


\subsection{Dlouhý světlovod}

Druhý systém, taktéž světlovod, se od prvního liší pouze v celkové delce, která je dvojnásobná:

\begin{quote}
    Jednoduchý zrcadlový světlovod se šířkou vstupu $2\ \mathrm{m}$ a šířkou výstupu $5\ \mathrm{m}$. Jeho celková délka činí $10\ \mathrm{m}$ a na vstupu do světlovodu je připevněn úsečkový zdroj o délce $1,95\ \mathrm{m}$.
\end{quote}

Odpovídající definice scény:

\begin{minipage}{\textwidth}\begin{quote}\begin{lstlisting}
Scene scene = Scene.create(
        // Horní část světlovodu
        Interactors.reflecting(
                Geometries.line(point(0, 1), point(10, 2))
        ),
        // Spodní část světlovodu
        Interactors.reflecting(
                Geometries.line(point(0, -1), point(10, -2))
        ),

        // Zdroj paprsků
        Emitters.line(
                point(0, 0),
                1.95f,
                100000
        )
);
EtendueApp.run(scene);
\end{lstlisting}\end{quote}\end{minipage}

Vygenerované vizualizace:

\singleimage{
    \image[scale=0.15]{vysledky_1b_scena.png}{Schéma dlouhého světlovodu}
}

\doubleimage{
    \image[scale=0.15]{vysledky_1b_etendue.png}{Pf-diagram na výstupu z dlouhého světlovodu}
}{
    \image[scale=0.15]{vysledky_1b_graf.png}{Rozložení svítivosti po průchodu dlouhým světlovodem}
}


\section{Parabolické reflektory}
\label{sec:parabolickereflektory}

\subsection{Parabolický reflektor s ohniskem ve středu vstupního otvoru}

Pokračujeme do kategorie parabolických reflektorů. Prvním takovým systémem je tento:

\begin{quote}
    Parabolický reflektor se vstupem šířky $2\ \mathrm{m}$ a délkou $5\ \mathrm{m}$. Parabola je umístěna tak, aby její ohnisko leželo na středu vstupního otvoru. Ve něm je také připevněn úsečkový zdroj o délce $1\ \mathrm{m}$.
\end{quote}

Tuto scénu jsme definovali následujícím způsobem. Poukažme na to, že jsou použity proměnné za účelem zjednodušení případné změny v zadání:

\begin{minipage}{\textwidth}\begin{quote}\begin{lstlisting}
// Délka reflektoru
float length = 5f;
// Poměr vzdáleností ohniska paraboly a vstupu reflektoru
// od vrcholu paraboly
float offsetRatio = 1f;


// Obecné vyjádření reflektoru pro výše zadané rozměry.
// Předpoklad je, že vstupní otvor se nachází na souřadnici
// x = 0 m a má velikost 2 m.

// Vypočtený parametr obecné rovnice paraboly
float param = (float) Math.sqrt(offsetRatio);
// Posunutí po ose x, aby reflektor začínal na x = 0
float offset = param / (2 * offsetRatio);
// Maximální hodnota, pro kterou počítáme tvar reflektoru
float max = (float) Math.sqrt( (length+offset) * 2*param );

Scene scene = Scene.create(
        // Horní část reflektoru
        Interactors.reflecting(
                Geometries.formula(
                        true,
                        -offset, 0,
                        1, max,
                        0.05f, x -> x*x / (2*param)
                )
        ),
        // Spodní část reflektoru
        Interactors.reflecting(
                Geometries.formula(
                        true,
                        -offset, 0,
                        -max, -1,
                        0.05f, x -> x*x / (2*param)
                )
        ),

        // Zdroj paprsků
        Emitters.line(
                point(0, 0),
                1f,
                100000
        )
);
EtendueApp.run(scene);
\end{lstlisting}\end{quote}\end{minipage}

Vygenerované vizualizace:

\singleimage{
    \image[scale=0.15]{vysledky_2a_scena.png}{Schéma parabolického reflektoru s ohniskem ve středu vstupního otvoru}
}

\doubleimage{
    \image[scale=0.15]{vysledky_2a_etendue.png}{Pf-diagram na výstupu z parabolického reflektoru s ohniskem ve středu vstupního otvoru}
}{
    \image[scale=0.15]{vysledky_2a_graf.png}{Rozložení svítivosti po průchodu parabolickým s ohniskem ve středu vstupního otvoru}
}


\subsection[Parabolický reflektor s ohniskem za vstupním otvorem]{Parabolický reflektor s ohniskem za vstupním otvorem (uvnitř těla reflektoru)}

Dalším systémem je podobný parabolický reflektor jako v předchozím případě, ohnisko paraboly je jen posunuto mírně doprava, zatímco ostatní podmínky zůstávají stejné:

\begin{quote}
    Parabolický reflektor se vstupem šířky $2\ \mathrm{m}$ a délkou $5\ \mathrm{m}$. Parabola je umístěna tak, aby její ohnisko leželo uvnitř těla reflektoru. Ve vstupním otvoru je připevněn úsečkový zdroj o délce $1\ \mathrm{m}$.
\end{quote}

Zdůrazněme, že v tomto kontextu nemá posun ohniska za následek posun celé paraboly, nýbrž v tomto případě je nutné celou rovnici paraboly přepočítat. Zde se nám vyplatí výše zmíněná definice pomocí několika proměnných, díky ní stačí změnit pouze jedno číslo:

\begin{minipage}{\textwidth}\begin{quote}\begin{lstlisting}
// Délka reflektoru
float length = 5f;
// Ohnisko unvitř reflektoru
float offsetRatio = 7 / 5f;

// ...
// Zbytek kódu je stejný
\end{lstlisting}\end{quote}\end{minipage}

Vygenerované vizualizace:

\singleimage{
    \image[scale=0.15]{vysledky_2b_scena.png}{Schéma parabolického reflektoru s ohniskem za vstupním otvorem}
}

\doubleimage{
    \image[scale=0.15]{vysledky_2b_etendue.png}{Pf-diagram na výstupu z parabolického reflektoru s ohniskem za vstupním otvorem}
}{
    \image[scale=0.15]{vysledky_2b_graf.png}{Rozložení svítivosti po průchodu parabolickým s ohniskem za vstupním otvorem}
}


\subsection[Parabolický reflektor s ohniskem před vstupním otvorem]{Parabolický reflektor s ohniskem před vstupním otvorem (mimo tělo reflektoru)}

Podobná úprava reflektoru jako v předchozím případě, ovšem ohnisko je posunuto doleva:

\begin{quote}
    Parabolický reflektor se vstupem šířky $2\ \mathrm{m}$ a délkou $5\ \mathrm{m}$. Parabola je umístěna tak, aby její ohnisko leželo nalevo od těla reflektoru. Ve vstupním otvoru je připevněn úsečkový zdroj o délce $1\ \mathrm{m}$.
\end{quote}

Odpovídající definice scény:

\begin{minipage}{\textwidth}\begin{quote}\begin{lstlisting}
// Délka reflektoru
float length = 5f;
// Ohnisko mimo reflektor
float offsetRatio = 3 / 4f;

// ...
// Zbytek kódu je stejný
\end{lstlisting}\end{quote}\end{minipage}

Vygenerované vizualizace:

\singleimage{
    \image[scale=0.15]{vysledky_2c_scena.png}{Schéma parabolického reflektoru s ohniskem před vstupním otvorem}
}

\doubleimage{
    \image[scale=0.15]{vysledky_2c_etendue.png}{Pf-diagram na výstupu z parabolického reflektoru s ohniskem před vstupním otvorem}
}{
    \image[scale=0.15]{vysledky_2c_graf.png}{Rozložení svítivosti po průchodu parabolickým reflektorem s ohniskem před vstupním otvorem}
}


