\chapter{Výsledky}

Někde se zmínit, že obrázky ve celé práci jsou generované pomocí mého programu.

\section{Světlovody}

\subsection{Krátký světlovod}

\imageblock{
    \singleimage{
        \image[scale=0.15]{vysledky_1a_scena.png}{Scéna 1A}
    }

    \doubleimage{
        \image[scale=0.15]{vysledky_1a_etendue.png}{Pf-diagram 1A}
    }{
        \image[scale=0.15]{vysledky_1a_graf.png}{Rozložení svítivosti 1A}
    }   
}


\subsection{Dlouhý světlovod}

\imageblock{
    \singleimage{
        \image[scale=0.15]{vysledky_1b_scena.png}{Scéna 1B}
    }

    \doubleimage{
        \image[scale=0.15]{vysledky_1b_etendue.png}{Pf-diagram 1B}
    }{
        \image[scale=0.15]{vysledky_1b_graf.png}{Rozložení svítivosti 1B}
    }
}


\section{Parabolické reflektory}

\subsection{Reflektor s ohniskem v místě vstupního otvoru}


\subsection{Reflektor s ohniskem před vstupním otvorem}


\subsection{Reflektor s ohniskem za vstupním otvorem}




\fixme{Vlastními slovy}

Udělal bych cca pět příkladů
1a) jednoduchý zrcadlový světlovod, šířka vstupu 2, šířka výstupu 10,
délka 10; délka lineárního svítidla na vstupu 2
1b) jako 1a, ale délka 20

2a) parabolický reflektor, šířka vstupu 2, délka reflektoru 10, parabola
tak, aby její ohnisko leželo ve středu vstupního otvoru (vzorec pro
parabolu kdyžtak dodám)
2b) jako 2a, ale ohnisko záměrně za vstupním otvorem
2c) jako 2a, ale ohnisko záměrně před vstupním otvorem
