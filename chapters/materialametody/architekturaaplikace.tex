\section{Architektura aplikace}

\fixme{Vlastními slovy}

\todo{Předpokládám, že tohle bude hodně v mé režii}

Popis by měl být přizpůsobený Vaší implementaci; níže je, jak bych to
implementoval já. Pokud to uznáte za vhodné, můžete jít až na
matematické vztahy, kterými něco počítáte, ale odkaz na literaturu je
dostatečný -- jde o standardní operace, které není nutné rozpitvávat.
Uznáte-li za vhodné přidat obrázky, přidejte.

Sledování paprsku (ray tracing) - ze zdroje světla se vyšle paprsek,
spočítá se průsečík se všemi objekty, vybere se nejbližší, vypočítá se
interakce (odraz, absorpce) a případně se ve sledování paprsku pokračuje
dál, dokud se nedosáhne limitního počtu interakcí (např. 1000), nebo
paprsek neopustí scénu (průsečík neexistuje).

Vizualizace chodu paprsků simulovaným systémem - tady asi jen napsat, že
k pochopení chování optického systému je užitečné vidět nejenom
abstraktní prostorovo-fázový diagram, ale i jednotlivé optické členy a
chod paprsků mezi nimi. A důležitá drobnost: pro výpočet
prostorovo-fázového diagramu potřebujete třeba statisíce paprsků, ale
pro vizualizaci chodu paprsků je vhodné nakreslit jich *maximálně* pár
set, jinak je z toho nepřehledná změť čar. Máte-li ve vizualizaci nějaký
implementačně zajímavý detail, klidně jej samozřejmě uveďte.

Výpočet prostorovo-fázového diagramu - výpočet průsečíku každého z
paprsků (jak polopřímek, tak úseček) s osou y zvoleného souřadného
systému Oxy; pokud existuje, zaznamenat průsečík + směr paprsku do
pomocného pole. Následně určit rozsah prostorovo-fázového diagramu,
rozdělit na pixely, do každého určit, kolik paprsků obsahuje.
Alternativně: rozsah prostorovo-fázového diagramu určit předem
(uživatelsky), aby byl během vizualizace rozsah pořád stejný.

Étendue: počet neprázdných pixelů. Tok: celkový počet paprsků v pomocném
poli průsečíků. Rozložení svítivosti: součet počtu paprsků v každém
diskrétním úhlu (šířky 1 pixel) prostorovo-fázového diagramu. Někde
hlásit, kolik paprsků z pomocného pole průsečíků není v
prostorovo-fázovém diagramu obsaženo.

Nevím, jestli něco obecného psát k obecné architektuře programu - to
musíte posoudit sám, jestli je tam něco mimořádně zajímavého.