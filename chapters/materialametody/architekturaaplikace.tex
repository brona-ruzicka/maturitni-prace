\section{Architektura a funkce aplikace}

V této kapitole nejprve obecně popíšeme postup použití aplikace a poté rozebereme jednotlivé fáze.

Zmiňme, že celý zdrojový kód programu uveden v je \emph{příloze č. 1} této práce. Pro potřeby této práce myslíme pod slovem \emph{třída} (anglicky \emph{class}) spíše \emph{Javovský} pojem \emph{typ} (anglicky \emph{type}). Ten zahrnuje jednak samotné \texttt{class}, ale i \texttt{interface} a \texttt{enum}.\src

\subsection{Typické použití aplikace}

Začneme tím, že uživatel definuje scénu, kterou chce prozkoumat, a následně spustí samotnou aplikaci.

V podstatě ihned poté se otevře počítačové okno zobrazující všechny prvky scény a zároveň se \emph{na pozadí} (tzn. uživatel nevidí žádnou indikaci a nemá žádnou přímou zpětnou vazbu) zahájí i simulace průchodu paprsků scénou (anglicky \emph{ray tracing}). Zatímco se čeká, než proběhne tato fáze, je možné a vhodné zkontrolovat, zda-li skutečně zadaná a zobrazená scéna odpovídá záměru.

Když aplikace dokončí simulaci paprsků, dokreslí je do okna se scénou. Kromě nich ale i na výchozí pozici odpovídající $ x = 0 $ zobrazí svislou zelenou čáru, která naznačuje polohu \emph{detekční přímky}. Tuto čáru lze posouvat doprava a doleva s pomocí myši.

V ten samý okamžik se ovšem také otevřou další dvě okna, jedno obsahující \emph{prostorovo-fázový diagram} a druhé \emph{graf rozložení svítivosti}. Vizualizace těchto grafů reagují na změny pozice \emph{detekční přímky} v okně se scénou.

\subsection{Zadání scény}
\label{sub:architekturaaplikace_zadanisceny}

Z důvodu jednoduchosti implementace a snížení komplexity vývoje se scéna do aplikace zadává přímo jako kód, tj. pomocí funkcí definujících jednotlivé součásti optického systému. Když uživatel nadefinuje scénu, je nutné program zkompilovat a spustit, což by se mohlo zdát jako složitý postup, ale např. v autorem použitém prostředí IntelliJ IDEA k tomu v dnešní době stačí stisknutí jednoho tlačítka.

Pro spuštění aplikace je nutné zavolat funkci \texttt{EtendueApp\#run}, která vyžaduje jediný argument, a to objekt popisující simulovanou scénu. Právě třída \texttt{EtendueApp} sdružuje všechny části tohoto programu, ať se týkají simulace nebo zobrazení. Jednotlivé součásti popisujeme podrobněji dále v této kapitole.

K definici optického systému využíváme třídu \texttt{Scene}, což je v podstatě jen množina členů simulovaného optického systému. Ty jsou popsány třídou \texttt{Member}. Přičemž je můžeme rozdělit do dvou kategorií, jednak to jsou zdroje světla (třída \texttt{Emitter}) a druhou kategorii tvoří opticky interagující prvky (třída \texttt{Interactor}).

V kódu definujeme scénu pomocí funkce \texttt{Scene\#create}, jejíž argumenty jsou instance výše zmíněné třídy \texttt{Member}:

\begin{minipage}{\textwidth}\begin{quote}\begin{lstlisting}
// Vytvoření scény
Scene scene = Scene.create(

        // Zde budou definovány jednotlivé členy této scény
);

// Samotné spuštění simulace
EtendueApp.run(scene);
\end{lstlisting}\end{quote}\end{minipage}

Doplňme, že v jazyce \emph{Java} je potřeba pro úspěšné spuštění programu doplnit ještě několik dalších neměnných instrukcí. Pro úplnost je v tomto případě uvedeme, ale v dalších výňatcích obsahujících definice scén je vynecháme:

\begin{minipage}{\textwidth}\begin{quote}\begin{lstlisting}
// Obsah souboru Main.java

// Tady vynecháváme ještě tzv. import statements,
// které se mění v závislosti na použitých třídách

public class Main {
    public static void main(String[] args) {
        // Zde budou výňatky začínat

        // Vytvoření scény
        Scene scene = Scene.create(
                // Definice jednotlivých členů
        );

        // Samotné spuštění simulace
        EtendueApp.run(scene);

        // Zde budou výňatky končit
    }
}
\end{lstlisting}\end{quote}\end{minipage}

\subsubsection{Světelné zdroje}

Jediným úkolem instance třídy \texttt{Emitter} je vytvořit jakýsi seznam paprsků, které daný objekt vysílá. V našem případě jde o množinu dvojic \emph{počáteční bod} -- \emph{směrový vektor}.

V aplikaci je definováno několik základních druhů světelných zdrojů. Se všemi z nich jsme se již v této práci setkali.

\paragraph{Jednoduchý zdroj} je definován bodem a vektorem. Vysílá právě jeden paprsek.

\begin{minipage}{\textwidth}\begin{quote}\begin{lstlisting}
// Funkce vytvářející jednoduchý zdroj
Emitters.single(
    // Počáteční bod
    point(1, 1),
    // Směrový vektor
    vector(1, -1)
)
\end{lstlisting}\end{quote}\end{minipage}

\singleimage{
    \image[scale=0.15]{architekturaaplikace_emitters_simple.png}{Příklad scény obsahující pouze jednoduchý zdroj světla}
}

Kromě již popsaných částí kódu si všimněme použití pomocných funkcí \texttt{point} a \texttt{vector}, které vytvářejí objekty reprezentující bod resp. vektor (v obou případech jde de facto o uspořádanou dvojici čísel). Tyto funkce se hodí zejména pro vyšší čitelnost a jednodušší porozumění definice z pohledu uživatele.

\paragraph{Bodový zdroj} je definován bodem a počtem paprsků. Vysílá daný počet paprsků rovnoměrně rozložený do všech směrů.

\begin{minipage}{\textwidth}\begin{quote}\begin{lstlisting}
// Funkce vytvářející bodový zdroj
Emitters.point(
    // Střed zdroje
    point(1, 1),
    // Počet paprsků
    10
)
\end{lstlisting}\end{quote}\end{minipage}

\singleimage{
    \image[scale=0.15]{architekturaaplikace_emitters_point.png}{Příklad scény obsahující pouze bodový zdroj světla}
}

\paragraph{Úsečkový zdroj} je definován středem, délkou a počtem paprsků. Je vždy orientován svisle a paprsky vysílá vždy směrem doprava. O vlastnostech tohoto typu zářiče jsme se již zmiňovali v kapitole \fullref{sub:analyzasystemu_vlastnostipfdiagramu}, ale pro jistotu zopakujme, že jej považujeme za \emph{lambertovský zářič}.\src

\begin{minipage}{\textwidth}\begin{quote}\begin{lstlisting}
// Funkce vytvářející úsečkový zdroj
Emitters.line(
    // Střed zdroje
    point(1, 0),
    // Délka zářiče
    2,
    // Počet paprsků
    10
)
\end{lstlisting}\end{quote}\end{minipage}

\singleimage{
    \image[scale=0.15]{architekturaaplikace_emitters_line.png}{Příklad scény obsahující pouze úsečkový zdroj světla}
}

Uživateli je samozřejmě umožněno definovat vlastní světelné zářiče. Např. vějířový zdroj (viz obr. \ref{fig:analyzasystemu_priklad_soucin.png}) je velmi jednoduchou úpravou bodového zdroje.


\subsubsection{Typy optických prvků}

Obecně instance třídy \texttt{Interactor} nesou dvě informace. Jednak samozřejmě svůj tvar, a pak také nějaký předpis, který říká, co se děje s dopadajícími paprsky.

Jak je vysvětleno již v kapitole \fullref{sec:vymezeniproblemu}, v této aplikaci jsme se omezili pouze na dva typy optických členů, \emph{dokonalé clony} a \emph{dokonalá zrcadla}. Funkce, které je vytvářejí, vyžadují pouze jeden argument, a tím je objekt popisující tvar. Různým vyjádřením tvaru se věnujeme v následující podkapitole.

\paragraph{Dokonalá clona} je definováno takto:

\begin{minipage}{\textwidth}\begin{quote}\begin{lstlisting}
// Funkce vytvářející dokonalou clonu
Interactors.absorbing(
    // Zde je definice tvaru
)
\end{lstlisting}\end{quote}\end{minipage}
    

\paragraph{Dokonalé zracadlo} je definováno takto:

\begin{minipage}{\textwidth}\begin{quote}\begin{lstlisting}
// Funkce vytvářející dokonalé zrcadlo
Interactors.reflecting(
    // Zde je definice tvaru
)
\end{lstlisting}\end{quote}\end{minipage}


\subsubsection{Geometrie optických prvků}




\subsubsection{Vybrané příklady scén}



\subsection{Simulace paprsků}
\label{sub:architekturaaplikace_simulacepaprsku}

\fixme{Vlastními slovy}

Sledování paprsku (ray tracing) - ze zdroje světla se vyšle paprsek,
spočítá se průsečík se všemi objekty, vybere se nejbližší, vypočítá se
interakce (odraz, absorpce) a případně se ve sledování paprsku pokračuje
dál, dokud se nedosáhne limitního počtu interakcí (např. 1000), nebo
paprsek neopustí scénu (průsečík neexistuje).

\todo{Zmínit trik s homogeními souřadnicemi}

\subsection{Vizualizace scény}

\fixme{Vlastními slovy}

Vizualizace chodu paprsků simulovaným systémem - tady asi jen napsat, že
k pochopení chování optického systému je užitečné vidět nejenom
abstraktní prostorovo-fázový diagram, ale i jednotlivé optické členy a
chod paprsků mezi nimi. A důležitá drobnost: pro výpočet
prostorovo-fázového diagramu potřebujete třeba statisíce paprsků, ale
pro vizualizaci chodu paprsků je vhodné nakreslit jich *maximálně* pár
set, jinak je z toho nepřehledná změť čar. Máte-li ve vizualizaci nějaký
implementačně zajímavý detail, klidně jej samozřejmě uveďte.

\subsection{Detekce paprsků}

\fixme{Vlastními slovy}

Výpočet průsečíku každého z
paprsků (jak polopřímek, tak úseček) s osou y zvoleného souřadného
systému Oxy; pokud existuje, zaznamenat průsečík + směr paprsku do
pomocného pole.

\todo{Zmínit optimalizaci triku s homogeními souřadnicemi}

\subsection{Vizualizace prostorovo-fázového diagramu}

\fixme{Vlastními slovy}

Následně určit rozsah prostorovo-fázového diagramu,
rozdělit na pixely, do každého určit, kolik paprsků obsahuje.
Alternativně: rozsah prostorovo-fázového diagramu určit předem
(uživatelsky), aby byl během vizualizace rozsah pořád stejný.

Étendue: počet neprázdných pixelů. Tok: celkový počet paprsků v pomocném
poli průsečíků. Rozložení svítivosti: součet počtu paprsků v každém
diskrétním úhlu (šířky 1 pixel) prostorovo-fázového diagramu. Někde
hlásit, kolik paprsků z pomocného pole průsečíků není v
prostorovo-fázovém diagramu obsaženo.

\subsection{Vizualizace grafu rozložení svítivosti}

\todo{}
