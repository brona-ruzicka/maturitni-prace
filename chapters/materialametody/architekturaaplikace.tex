\section{Architektura a funkce aplikace}

V této kapitole nejprve obecně popíšeme postup použití aplikace a poté rozebereme jednotlivé součásti programu.

Zmiňme, že celý zdrojový kód programu je uveden v \emph{příloze č. 1} této práce. Pro potřeby této práce používáme termín \emph{třída} (anglicky \emph{class}) spíše pro \emph{Javovský} pojem \emph{typ} (anglicky \emph{type}). Ten zahrnuje jednak samotné \texttt{class}, ale i \texttt{interface} a \texttt{enum}.

\subsection{Typické použití aplikace}

Začneme tím, že uživatel definuje scénu, kterou chce prozkoumat, a následně spustí samotnou aplikaci.

V podstatě ihned poté se otevře hlavní okno aplikace zobrazující všechny prvky scény a zároveň se \emph{na pozadí} (tzn. uživatel nevidí žádnou indikaci a nemá žádnou přímou zpětnou vazbu) zahájí i simulace průchodu paprsků scénou (anglicky \emph{ray tracing}). Během této fáze (zatímco probíhá simulace) je možné a vhodné zkontrolovat, zda-li skutečně zadaná a zobrazená scéna odpovídá záměru.

V momentě kdy aplikace dokončí simulaci paprsků, dokreslí je do okna se scénou. Kromě nich ale i na pozici odpovídající $ x = 0 $ zobrazí svislou zelenou čáru, která určuje polohu \emph{detekční přímky}. Touto čárou lze posouvat doprava a doleva s pomocí myši.

V ten samý okamžik se otevřou ještě další dvě okna, jedno obsahující \emph{prostorovo-fázový diagram} a druhé \emph{graf rozložení svítivosti}. Vizualizace těchto grafů reagují na změ\-ny pozice \emph{detekční přímky} v okně se scénou.

\subsection{Zadání scény}
\label{sub:architekturaaplikace_zadanisceny}

Z důvodu jednoduchosti implementace a snížení komplexity vývoje se scéna do aplikace zadává přímo jako kód, tj. pomocí funkcí definujících jednotlivé součásti optického systému. Když uživatel nadefinuje scénu, je nutné program zkompilovat a spustit, což by se mohlo zdát jako složitý postup, ale např. v autorem použitém prostředí IntelliJ IDEA k tomu v dnešní době stačí stisknutí jednoho tlačítka.

Pro spuštění aplikace je nutné zavolat funkci \texttt{EtendueApp.run()}, která vyžaduje jediný argument, a to objekt popisující simulovanou scénu. Právě třída \texttt{EtendueApp} propojuje všechny části aplikace, ať se týkají simulace nebo zobrazení. Podrobněji je popisujeme dále v této kapitole.

K definici optického systému využíváme třídu \texttt{Scene}, což je v podstatě jen množina členů simulovaného optického systému. Ty jsou popsány třídou \texttt{Member}, přičemž je můžeme rozdělit do dvou kategorií, na zdroje světla (třída \texttt{Emitter}) a na opticky interagující prvky (třída \texttt{Interactor}).

V kódu definujeme scénu pomocí funkce \texttt{Scene\#create}, jejíž argumenty jsou instance výše zmíněné třídy \texttt{Member}:

\begin{minipage}{\textwidth}\begin{quote}\begin{lstlisting}
// Vytvoření scény
Scene scene = Scene.create(

        // Zde budou definovány jednotlivé členy této scény

);

// Samotné spuštění simulace
EtendueApp.run(scene);
\end{lstlisting}\end{quote}\end{minipage}

Doplňme, že v jazyce \emph{Java} je potřeba pro úspěšné spuštění programu doplnit ještě několik dalších neměnných instrukcí. Pro úplnost je v tomto případě uvedeme, ale v dalších výňatcích obsahujících definice scén je vynecháme:

\begin{minipage}{\textwidth}\begin{quote}\begin{lstlisting}
// Obsah souboru Main.java

// Tady vynecháváme ještě tzv. import statements,
// které se mění v závislosti na použitých třídách

public class Main {
    public static void main(String[] args) {

        // Vytvoření scény
        Scene scene = Scene.create(
                // Definice jednotlivých členů
        );

        // Samotné spuštění simulace
        EtendueApp.run(scene);

    }
}
\end{lstlisting}\end{quote}\end{minipage}

\subsubsection{Světelné zdroje}

Jediným úkolem instance třídy \texttt{Emitter} je vytvořit jakýsi seznam paprsků, které daný objekt vysílá. V našem případě jde o množinu dvojic \emph{počáteční bod} -- \emph{směrový vektor}.

V aplikaci je definováno několik základních druhů světelných zdrojů. Se všemi z nich jsme se již v této práci setkali.

\paragraph{Jednoduchý zdroj} je definován bodem a vektorem. Vysílá právě jeden paprsek.

\begin{minipage}{\textwidth}\begin{quote}\begin{lstlisting}
// Funkce vytvářející jednoduchý zdroj
Emitters.single(
        // Počáteční bod
        point(1, 1),
        // Směrový vektor
        vector(1, -1)
)
\end{lstlisting}\end{quote}\end{minipage}

\singleimage{
    \image[scale=0.15]{architekturaaplikace_emitters_simple.png}{Příklad scény obsahující pouze jednoduchý zdroj světla}
}

Kromě již popsaných částí kódu si všimněme použití pomocných funkcí \texttt{point} a \texttt{vector}, které vytvářejí objekty reprezentující bod resp. vektor (v obou případech jde de facto o uspořádanou dvojici čísel). Tyto funkce se hodí zejména pro lepší čitelnost a jednodušší porozumění definice z pohledu uživatele.

\paragraph{Bodový zdroj} je definován bodem a počtem paprsků. Vysílá daný počet paprsků rovnoměrně rozložený do všech směrů.

\begin{minipage}{\textwidth}\begin{quote}\begin{lstlisting}
// Funkce vytvářející bodový zdroj
Emitters.point(
        // Střed zdroje
        point(1, 1),
        // Počet paprsků
        10
)
\end{lstlisting}\end{quote}\end{minipage}

\singleimage{
    \image[scale=0.15]{architekturaaplikace_emitters_point.png}{Příklad scény obsahující pouze bodový zdroj světla}
}

\paragraph{Úsečkový zdroj} je definován středem, délkou a počtem paprsků. Je vždy orientován svisle a paprsky vysílá vždy směrem doprava. O vlastnostech tohoto typu zářiče jsme se již zmiňovali v kapitole \fullref{sub:analyzasystemu_vlastnostipfdiagramu}, ale pro jistotu zopakujme, že jej považujeme za \emph{lambertovský zářič}.\src

\begin{minipage}{\textwidth}\begin{quote}\begin{lstlisting}
// Funkce vytvářející úsečkový zdroj
Emitters.line(
        // Střed zdroje
        point(1, 0),
        // Délka zářiče
        2,
        // Počet paprsků
        10
)
\end{lstlisting}\end{quote}\end{minipage}

\singleimage{
    \image[scale=0.15]{architekturaaplikace_emitters_line.png}{Příklad scény obsahující pouze úsečkový zdroj světla}
}

Uživateli je samozřejmě umožněno definovat vlastní světelné zářiče. Např. vějířový zdroj (viz obr. \ref{fig:analyzasystemu_priklad_soucin.png}) je velmi jednoduchou úpravou bodového zdroje.


\subsubsection{Typy optických prvků}

Obecně instance třídy \texttt{Interactor} nesou dvě informace. Jednak samozřejmě svůj tvar, a pak také nějaký předpis, který říká, co se děje s dopadajícími paprsky.

Jak je vysvětleno již v kapitole \fullref{sec:vymezeniproblemu}, v této aplikaci jsme se omezili pouze na dva typy optických členů, \emph{dokonalé clony} a \emph{dokonalá zrcadla}. Funkce, které je vytvářejí, vyžadují pouze jeden argument, a tím je objekt popisující tvar. Různým vyjádřením tvaru se věnujeme v následující podkapitole.

\paragraph{Dokonalá clona} se vytváří takto:

\begin{minipage}{\textwidth}\begin{quote}\begin{lstlisting}
// Funkce vytvářející dokonalou clonu
Interactors.absorbing(
        // Zde je definice tvaru
)
\end{lstlisting}\end{quote}\end{minipage}
    

\paragraph{Dokonalé zracadlo} se vytváří následujícím způsobem:

\begin{minipage}{\textwidth}\begin{quote}\begin{lstlisting}
// Funkce vytvářející dokonalé zrcadlo
Interactors.reflecting(
        // Zde je definice tvaru
)
\end{lstlisting}\end{quote}\end{minipage}


\subsubsection{Geometrie optických prvků}

Aplikace podporuje několik různých typů geometrií optických prvků. My se zde zmíníme o dvou.

\paragraph{Úsečka} je definována pomocí dvou bodů:

\begin{minipage}{\textwidth}\begin{quote}\begin{lstlisting}
// Funkce vytvářející úsečkovou geometrii
Geometries.line(
        // Počáteční bod
        point(0,1),
        // Koncový bod
        point(5,2)
)
\end{lstlisting}\end{quote}\end{minipage}

\singleimage{
    \image[scale=0.15]{architekturaaplikace_line.png}{Příklad scény obsahující úsečkovou geometrii}
}


\paragraph{Geometrie podle matematické fuknce} je zadávána poněkud složitěji. Parametry jsou popsány přímo v následujícím kódu.


\begin{minipage}{\textwidth}\begin{quote}\begin{lstlisting}
// Funkce vytvářející geometrii podle funkce
Geometries.formula(
        // Vyjadřujeme x v závislosti na y, nebo obráceně
        true,
        // Počáteční bod, vyjadřuje, kde se promítne bod [0,0]
        point(1, 2),
        // Výchozí hodnota parametru
        -3,
        // Koncová hodnota parametru
        3,
        // Krok parametru
        0.1f,
        // Samotná funkce, do které je dosazován parametr
        x -> x*x / 2
)
\end{lstlisting}\end{quote}\end{minipage}

\singleimage{
    \image[scale=0.15]{architekturaaplikace_parabola.png}{Příklad scény obsahující aproximaci paraboly}
}

\subsubsection{Příklad kompletní scény}

Jako příklad nám poslouží jednoduchý světlovod, zmíněný v kapitole \fullref{sec:vymezeniproblemu}.

\begin{minipage}{\textwidth}\begin{quote}\begin{lstlisting}
Scene scene = Scene.create(
        // Horní část
        Interactors.reflecting(
                Geometries.line(point(0, 1), point(5, 2))
        ),
        // Spodní část
        Interactors.reflecting(
                Geometries.line(point(0, -1), point(5, -2))
        ),
        // Zdroj světla
        Emitters.line(point(0, 0), 2, 10)
);

EtendueApp.run(scene);
\end{lstlisting}\end{quote}\end{minipage}

S touto soustavou pracujeme i v následujících podkapitolách.


\subsection{Simulace paprsků}
\label{sub:architekturaaplikace_simulacepaprsku}

Simulace průchodu paprsků scénou se provede pouze jednou, hned po spuštění programu. Výsledek je uložen do paměti pro pozdější použití, tedy zejména výpočet průsečíků s detekční přímkou.

V naší aplikaci implementujeme jednoduchý \emph{ray tracing} algoritmus. Nejprve se vyšle paprsek ze zdroje světla. Poté se spočítá jeho průsečík se všemi objekty ve scéně a vybere se ten nejbližší. Následně se vyhodnotí interakce s tímto objektem (odraz nebo absorpce). A pokud se paprsek odráží, postup se opakuje. Sledování paprsku pokračuje, dokud se nedosáhne limitu počtu interakcí (v našem případě 1000), nebo dokud paprsek neopustí scénu, tzn. žádný další průsečík neexistuje.

Zajímavostí je, že pro výpočet průsečíku používáme tzv. \emph{homogenní souřadnice} bodu a vzorec pro průsečík dvou přímek definovaný sérií vektorových součinů, jak popsal \textcite{skala2008intersection}. Pro získaný bod poté jen ověříme, že se nachází na správné polopřímce resp. úsečce.

Dále můžeme uvést, že v zájmu zkrácení celkového času potřebného k výpočtu používáme tzv. \emph{multithreading}, tj. způsob jak spustit více výpočtů paralelně.

\subsection{Vizualizace scény}

Vizualizace scény proběhne za celou dobu běhu programu pouze dvakrát. Nejprve při spuštění, to jsou zobrazeny pouze jednotlivé členy scény.

\singleimage{
        \image[scale=1]{architekturaaplikace_preview.png}{Příklad okna schématu scény, ještě bez nasimulovaných paprsků}
}

Podruhé jsou vykresleny opět všechny členy scény, ale tentokrát společně s několika náhodně vybranými paprsky (zobrazování všech simulovaných paprsků by bylo nepřehledné). Tato vizualizace se uloží do paměti programu. Když uživatel posune detekční přímku, do okna se scénou už vykreslí jen tento předvygenerovaný obrázek (nemusíme kreslit znova každý paprsek) a navrh se pouze dokreslí svislá čára znázorňující detekční přímku.

\singleimage{
        \image[scale=1]{architekturaaplikace_spaprsky.png}{Příklad okna schématu scény, s nasimulovanými paprsky a detekční přímkou}
}

Pro zobrazení scény jsme zvolili následující barvy: červená pro zdroje a paprsky, modrá pro zrcadla a tmavě šedá pro clony. Dále zelenou barvou je vykreslena detekční přímka a černými a světle šedými čárami je vyznačena vztažná soustava.

Zároveň uveďme, že program sám přizpůsobuje, která část scény se zobrazí a i v jakém měřítku, a to tak, aby všechny prvky scény byly vždy kompletně viditelné. Dále mřížka a její popisky se taktéž automaticky uzpůsobí zvolenému přiblížení.

\subsection{Detekce paprsků}

Výpočet průsečíků paprsků s detekční přímkou proběhne pokaždé, když se změní její poloha. Přičemž k výpočtu jsou užita uložená data o chodu paprsků z fáze sledování paprsků. Používáme stejný vzorec jako v podkapitole \fullref{sub:architekturaaplikace_simulacepaprsku}. Jen mírně zjednodušený, protože víme, že jedna z přímek bude vždy rovnoběžná s osou $y$.

Z důvodu zajištění plynulého chodu aplikace z pohledu uživatele, probíhá i tento výpočet \emph{na pozadí}.


\subsection{Vizualizace prostorovo-fázového diagramu}

Ihned po vypočtení průsečíků paprsků s detekční přímkou se spustí část zobrazující prostorovo-fázový diagram.

Nejprve si plochu okna pf-diagramu rozdělí na mřížku. A poté pro každý paprsek vypočte, do kterého políčka spadá. Nakonec pro každé políčko vykreslí čtvereček vybarvený odstínem červené odpovídající počtu paprsků v daném políčku.

V tomtéž okně se také zobrazí další dvě veličiny. Jednak étendue, které se vypočte jako plocha neprázdných políček pf-diagramu. A pak průměrný počet paprsků v jednom políčku (položka \emph{Average}).

\singleimage{
        \image[scale=1]{architekturaaplikace_etendue.png}{Příklad okna prostorovo-fázového diagramu}
}

\subsection{Vizualizace grafu rozložení svítivosti}

Současně se zobrazením pf-diagramu se spustí i vykreslení grafu rozložení svítivosti. Postup je obdobný jako vizualizace pf-diagramu.

Opět se nejprve vytvoří datová struktura jako podklad pro vykreslení grafu, ovšem tentokrát je pouze jednorozměrná. Pro každý paprsek se vypočte, do kterého intervalu spadá. Na základě toho pak se zobrazí čára spojující vypočtené hodnoty.

Poznamenejme, že v tomto grafu zobrazujeme čáry dvě, jednu červenou a jednu fialovou. Červená znázorňuje počty paprsků mířící ve směru $+x$ a fialová směr $-x$.

Dále v okně vypíšeme ještě informaci o celkovém počtu paprsků (položka \emph{Total}). A také maximální hodnotu počtu paprsků v jednom intervalu (položka \emph{Maximum}).

\singleimage{
        \image[scale=1]{architekturaaplikace_graf.png}{Příklad okna grafu rozložení svítivosti}
}
