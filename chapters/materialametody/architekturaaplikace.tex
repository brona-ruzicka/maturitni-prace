\section{Architektura a funkce aplikace}

V této kapitole nejprve obecně popíšeme postup použití aplikace a poté rozebereme jednotlivé fáze. Celý proces popíšeme na příkladu světlometu uvedeném v kapitole \fullref{sec:vymezeniproblemu}.

\subsection{Typické použití aplikace}

Začneme tím, že uživatel definuje scénu, kterou chce prozkoumat, a následně spustí samotnou aplikaci.

V podstatě ihned poté se otevře počítačové okno zobrazující všechny prvky scény a zároveň se \emph{na pozadí} (tzn. uživatel nevidí žádnou indikaci a nemá žádnou přímou zpětnou vazbu) zahájí i simulace průchodu paprsků scénou (anglicky \emph{ray tracing}). Zatímco se čeká, než proběhne tato fáze, je možné a vhodné zkontrolovat, zda-li skutečně zadaná a zobrazená scéna odpovídá záměru.

Když program dokončí simulaci paprsků, dokreslí je do okna se scénou. Kromě nich ale i na výchozí pozici odpovídající $ x = 0 $ zobrazí svislou zelenou čáru, která naznačuje polohu \emph{detekční přímku}. Tuto čáru lze posouvat doprava a doleva s pomocí myši.

V ten samý okamžik se ovšem také otevřou další dvě okna, jedno obsahující \emph{prostorovo-fázový diagram} a druhé \emph{graf rozložení svítivosti}. Vizualizace těchto grafů reagují na změny pozice \emph{detekční přímky} v okně se scénou.

\subsection{Zadání scény}

\todo{Popsat možnosti a některé základní geometrie}

\subsection{Simulace paprsků}

\fixme{Vlastními slovy}

Sledování paprsku (ray tracing) - ze zdroje světla se vyšle paprsek,
spočítá se průsečík se všemi objekty, vybere se nejbližší, vypočítá se
interakce (odraz, absorpce) a případně se ve sledování paprsku pokračuje
dál, dokud se nedosáhne limitního počtu interakcí (např. 1000), nebo
paprsek neopustí scénu (průsečík neexistuje).

\todo{Zmínit trik s homogeními souřadnicemi}

\subsection{Vizualizace scény}

\fixme{Vlastními slovy}

Vizualizace chodu paprsků simulovaným systémem - tady asi jen napsat, že
k pochopení chování optického systému je užitečné vidět nejenom
abstraktní prostorovo-fázový diagram, ale i jednotlivé optické členy a
chod paprsků mezi nimi. A důležitá drobnost: pro výpočet
prostorovo-fázového diagramu potřebujete třeba statisíce paprsků, ale
pro vizualizaci chodu paprsků je vhodné nakreslit jich *maximálně* pár
set, jinak je z toho nepřehledná změť čar. Máte-li ve vizualizaci nějaký
implementačně zajímavý detail, klidně jej samozřejmě uveďte.

\subsection{Detekce paprsků}

\fixme{Vlastními slovy}

Výpočet průsečíku každého z
paprsků (jak polopřímek, tak úseček) s osou y zvoleného souřadného
systému Oxy; pokud existuje, zaznamenat průsečík + směr paprsku do
pomocného pole.

\todo{Zmínit optimalizaci triku s homogeními souřadnicemi}

\subsection{Vizualizace prostorovo-fázového diagramu}

\fixme{Vlastními slovy}

Následně určit rozsah prostorovo-fázového diagramu,
rozdělit na pixely, do každého určit, kolik paprsků obsahuje.
Alternativně: rozsah prostorovo-fázového diagramu určit předem
(uživatelsky), aby byl během vizualizace rozsah pořád stejný.

Étendue: počet neprázdných pixelů. Tok: celkový počet paprsků v pomocném
poli průsečíků. Rozložení svítivosti: součet počtu paprsků v každém
diskrétním úhlu (šířky 1 pixel) prostorovo-fázového diagramu. Někde
hlásit, kolik paprsků z pomocného pole průsečíků není v
prostorovo-fázovém diagramu obsaženo.

\subsection{Vizualizace grafu rozložení svítivosti}

\todo{}
