\section{Podporované členy scény}
\label{sec:podporovaneoptickeprvky}

V této kapitole se věnujeme jednotlivým podporovaným členům scény, tedy de facto implementacím třídy \texttt{Member} (viz kapitola \fullref{sub:architekturaaplikace_zadanisceny}).

\subsection{Světelné zdroje}

Jediným úkolem instance třídy \texttt{Emitter} je vytvořit jakýsi seznam paprsků, které daný objekt vysílá. V našem případě jde o množinu dvojic \emph{počáteční bod} -- \emph{směrový vektor}.

V aplikaci je definováno několik základních druhů světelných zdrojů. Se všemi z nich jsme se již v této práci setkali.

\paragraph{Jednoduchý zdroj} je definován bodem a vektorem. Vysílá právě jeden paprsek.

\begin{minipage}{\textwidth}\begin{quote}\begin{lstlisting}
// Funkce vytvářející jednoduchý zdroj
Emitters.single(
        // Počáteční bod
        point(1, 1),
        // Směrový vektor
        vector(1, -1)
)
\end{lstlisting}\end{quote}\end{minipage}

\singleimage{
    \image[scale=0.15]{architekturaaplikace_emitters_simple.png}{Příklad scény obsahující pouze jednoduchý zdroj světla}
}

Kromě již popsaných částí kódu si všimněme použití pomocných funkcí \texttt{point} a \texttt{vector}, které vytvářejí objekty reprezentující bod resp. vektor (v obou případech jde de facto o uspořádanou dvojici čísel). Tyto funkce se hodí zejména pro lepší čitelnost a jednodušší porozumění definice z pohledu uživatele.

\paragraph{Bodový zdroj} je definován bodem a počtem paprsků. Vysílá daný počet paprsků rovnoměrně rozložený do všech směrů.

\begin{minipage}{\textwidth}\begin{quote}\begin{lstlisting}
// Funkce vytvářející bodový zdroj
Emitters.point(
        // Střed zdroje
        point(1, 1),
        // Počet paprsků
        10
)
\end{lstlisting}\end{quote}\end{minipage}

\singleimage{
    \image[scale=0.15]{architekturaaplikace_emitters_point.png}{Příklad scény obsahující pouze bodový zdroj světla}
}

\paragraph{Úsečkový zdroj} je definován středem, délkou a počtem paprsků. Je vždy orientován svisle a paprsky vysílá vždy směrem doprava. O vlastnostech tohoto typu zářiče jsme se již zmiňovali v kapitole \fullref{sub:analyzasystemu_vlastnostipfdiagramu}, ale pro jistotu zopakujme, že jej považujeme za \emph{lambertovský zářič}. \parencite{fotometrie}

\begin{minipage}{\textwidth}\begin{quote}\begin{lstlisting}
// Funkce vytvářející úsečkový zdroj
Emitters.line(
        // Střed zdroje
        point(1, 0),
        // Délka zářiče
        2,
        // Počet paprsků
        10
)
\end{lstlisting}\end{quote}\end{minipage}

\singleimage{
    \image[scale=0.15]{architekturaaplikace_emitters_line.png}{Příklad scény obsahující pouze úsečkový zdroj světla}
}

Uživateli je samozřejmě umožněno definovat vlastní světelné zářiče. Např. vějířový zdroj (viz obr. \ref{fig:analyzasystemu_priklad_soucin.png}) je velmi jednoduchou úpravou bodového zdroje.


\subsection{Typy optických prvků}

Obecně instance třídy \texttt{Interactor} nesou dvě informace. Jednak samozřejmě svůj tvar, a pak také nějaký předpis, který říká, co se děje s dopadajícími paprsky.

Jak je vysvětleno již v kapitole \fullref{sec:vymezeniproblemu}, v této aplikaci jsme se omezili pouze na dva typy optických členů, \emph{dokonalé clony} a \emph{dokonalá zrcadla}. Funkce, které je vytvářejí, vyžadují pouze jeden argument, a tím je objekt popisující tvar. Různým vyjádřením tvaru se věnujeme v následující podkapitole.

\paragraph{Dokonalá clona} se vytváří takto:

\begin{minipage}{\textwidth}\begin{quote}\begin{lstlisting}
// Funkce vytvářející dokonalou clonu
Interactors.absorbing(
        // Zde je definice tvaru
)
\end{lstlisting}\end{quote}\end{minipage}
    

\paragraph{Dokonalé zracadlo} se vytváří následujícím způsobem:

\begin{minipage}{\textwidth}\begin{quote}\begin{lstlisting}
// Funkce vytvářející dokonalé zrcadlo
Interactors.reflecting(
        // Zde je definice tvaru
)
\end{lstlisting}\end{quote}\end{minipage}


\subsection{Geometrie optických prvků}

Aplikace podporuje několik různých typů geometrií optických prvků. My se zde zmíníme o dvou.

\paragraph{Úsečka} je definována pomocí dvou bodů:

\begin{minipage}{\textwidth}\begin{quote}\begin{lstlisting}
// Funkce vytvářející úsečkovou geometrii
Geometries.line(
        // Počáteční bod
        point(0,1),
        // Koncový bod
        point(5,2)
)
\end{lstlisting}\end{quote}\end{minipage}

\singleimage{
    \image[scale=0.15]{architekturaaplikace_line.png}{Příklad scény obsahující úsečkovou geometrii}
}


\paragraph{Geometrie podle matematické fuknce} je zadávána poněkud složitěji. Parametry jsou popsány přímo v následujícím kódu.


\begin{minipage}{\textwidth}\begin{quote}\begin{lstlisting}
// Funkce vytvářející geometrii podle funkce
Geometries.formula(
        // Vyjadřujeme x v závislosti na y, nebo obráceně
        true,
        // Počáteční bod, vyjadřuje, kde se promítne bod [0,0]
        point(1, 2),
        // Výchozí hodnota parametru
        -3,
        // Koncová hodnota parametru
        3,
        // Krok parametru
        0.1f,
        // Samotná funkce, do které je dosazován parametr
        x -> x*x / 2
)
\end{lstlisting}\end{quote}\end{minipage}

\singleimage{
    \image[scale=0.15]{architekturaaplikace_parabola.png}{Příklad scény obsahující aproximaci paraboly}
}