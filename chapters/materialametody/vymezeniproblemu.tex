\section{Vymezení problému}

Smyslem této práce je pouze demonstrovat jistý koncept. Zaměříme proto jen na úplné jádro problému, a pokud se výsledek prokáže jako užitečný, můžeme uvažovat o nějakém zobecnění. Připomeňme si, že cílem je jednak samozřejmě pochopení celkové funkce navrhovaného systému, ovšem neméně nás zajímají i detaily o vystupujícím světle.

Příkladem, na kterém si ukážeme postupné zjednodušení, budiž tento jednoduchý světlovod. Bude mít tvar \emph{pláště komolého jehlanu (případně hranolu) s odříznutými podstavami}. Namísto horní podstavy umístíme \emph{plošný zdroj světla}. Vysílané paprsky se z něj šíří skrz kužel, a to jednak přímo, ale i s pomocí odrazů od stěn.

V prvním kroku se omezíme pouze na podélný řez této soustavy, jinými slovy budeme se pohybovat pouze ve 2D. Zdrojem světla bude nyní jen úsečka a celý světlovod se zjednoduší na dvě úsečky reprezentující zrcadla:

\image[scale=0.7]{vymezeniproblemu_priklad.png}{Jednoduchý 2D světlovod tvořený dvěma zrcadly (modrá) s úsečkovým zdrojem světla a několika znázorněnými paprsky (červená)}

V návaznosti je nutné podotknout, že výsledky simulace ve 2D se mohou v určitých směrech zásadně lišit od těch pro 3D. Jako jeden z rozdílů můžeme uvést fyzikální veličinu osvětlenost ( $ [\ E\ ] = 1 \ \frac{lm}{m^2} $ ). Ve třech prostorových dimenzích se snižuje s druhou mocninou vzdálenosti (podle \emph{zákona převrácených čtverců}, anglicky \emph{inverse-square law}), zatímco její dvourozměrný protějšek se bude zmenšovat pouze lineárně.

Zároveň zdůrazněme, že nebudeme nahrazovat \emph{plošný zdroj} světla \emph{bodovým}. Tato metoda je velmi typická pro zobrazovací optiku, ale v oblasti nezobrazovací optiky se příliš nepoužívá. Hlavním důvodem je velikost světelných zdrojů, která bývá v navrhovaných systémech zpravidla nezanedbatelná vůči velikosti celé soustavy.

Dále se omezíme na studium makroskopických soustav (svítilna, světlomet). V takto velkých systémech se v podstatě neprojevují (resp. projevují, ale zanedbatelně) \emph{vlnové vlastnosti světla}, které proto ze simulace můžeme vypustit. Ve výsledku tedy budeme na světlo nahlížet pouze z pohledu \emph{paprskové optiky}. To by nebylo možné např. při zkoumání optických vláken, které se svými rozměry více přibližují vlnové délce použitého světla.

Budeme předpokládat, že všechny paprsky mají stejný, \emph{elementární}, optický výkon. Optický výkon celého světelného zdroje je pak jasně dán pouhým počtem paprsků, které vyzařuje.

Pro jednoduchost budeme v demonstračním programu pracovat pouze se dvěma typy interakcí mezi systémem a paprskem, \emph{dokonalým zrcadlovým odrazem} a \emph{dokonalou absorpci}. Pro praktické využití by se hodilo zahrnout i \emph{lom světla}, \emph{rozptyl} (odraz od matného povrchu), \emph{částečný odraz} aj. Zanedbáme též závislost zmíněných jevů na polarizaci nebo vlnové délce (barvě) světla.

\image[scale=0.7]{vymezeniproblemu_reflektor.png}{Příklad dokonalého odrazu na zrcadle (modrá) s bodovým zdrojem světla a několika znázorněnými paprsky (červená)}

\image[scale=0.7]{vymezeniproblemu_absorber.png}{Příklad dokonalé absorbce na cloně (šedá) s bodovým zdrojem světla a několika znázorněnými paprsky (červená)}

Reálné optické prvky bývají ve většině případů tvarované podle nejrůznějších matematických předpisů (např. často $y = k \cdot x^2$). Ovšem jelikož výpočty s mnoha typy křivek a funkcí nebývají vždy snadné, budeme připouštět pouze aproximaci těchto tvarů pomocí lomených čar (série mnoha krátkých úseček). Při budoucím použití bylo vhodné podporovat alespoň některé hojně používané útvary (\emph{kvadratická funkce}, \emph{bezierovy křivky} aj.).

\image[scale=0.7]{vymezeniproblemu_aproximace.png}{Aproximace zrcadla ve tvaru kvadratické funkce (modrá) s úsečkovým zdrojem světla a několika znázorněnými paprsky (červená)}


Podstatný rozdíl je, že v nezobrazovací systémech zpravidla není zřejmé, jak přesně bude který paprsek postupovat optickou soustavou. Oproti tomu v rámci zobrazovacích optiky je to velmi jednoduše vyvoditelné. Jako příklad si vezměme teleskop se dvěma čočkami. U něj je na první pohled jasné, že paprsek světla nejdříve projde přes přední, a pak zadní plochu první čočky, a poté to samé pro čočku druhou. Zejména díky této znalosti je analýza i návrh zobrazovacích soustav podstatně jednodušší, než je tomu u systémů nezobrazovacích.
