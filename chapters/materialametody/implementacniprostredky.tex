\section{Implementační prostředky}

Vzhledem k zejména demonstrativní funkci vytvářené aplikace, přihlíželo se při výběru implementačních prostředků ke dvěma hlavním faktorům.

Jednak hrála roli vhodnost daného prostředku pro samotný úkol. Např. programovací jazyk hojně používaný na serverech se nemusí hodit pro vytváření uživatelských aplikací.

Za druhé pak rozhodovala také předpokládaná jednoduchost a rychlost implementace z pohledu autora práce. Tzn. bylo vhodné používat prostředky, se kterými se již setkal, a nemusel se tak kromě samotné implementace ještě seznamovat s novým prostředím.

\subsection{Programovací jazyk}

Pro vývoj aplikace byl zvolen programovací jazyk \emph{Java}.\src Jedním z důvodů je i to, že autor práce s tímto programovacím jazykem už v minulosti pracoval.

Dalším bodem je, že standardní \emph{SDK} (zkr. \emph{software development kit}, volně česky \emph{standardní knihovna}) tohoto jazyka obsahuje poměrně dobře popsaně \emph{API} (zkr. \emph{application programming interface}, česky \emph{rozhraní pro programování aplikací}), které programům umožňuje jednoduché vytváření \emph{GUI} (zkr. \emph{graphical user interface}, česky \emph{grafické uživatelské rozhraní}). To zahrnuje zobrazování počítačových oken a různé způsoby jak jejich obsah definovat a vykreslovat. Zmíněné \emph{API} se v \emph{Javě} nazývá \emph{Java Swing}.\src

Pro produkční software by byla samozřejmě důležitá především rychlost simulace. Z tohoto hlediska by bylo vhodnější použít např. programovací jazyky \emph{C++} nebo \emph{Rust}, které jsou na rozdíl od \emph{Javy} kompilované přímo do strojového kódu, a proto jsou schopny jednoduché aritmetické výpočty provádět podstatně rychleji.

\subsection{Vývojové prostředí}

Vývoj aplikace probíhal v prostředí \emph{IntelliJ IDEA} od české firmy \emph{JetBrains}. Pro kompilaci a správu závislostí (anglicky \emph{dependency management}, tj. především externí knihovny) byl použit program \emph{Maven}. Důvodem byla čistě jen autova znalost těchto nástrojů. 
