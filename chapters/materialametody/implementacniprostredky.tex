\section{Implementační prostředky}

Vzhledem k zejména demonstrativní funkci vytvářené aplikace, přihlíželo se při výběru implementačních prostředků ke dvěma hlavním faktorům.

Jednak to byla vhodnost daného prostředku pro samotný úkol. Např. pro aplikace provádějící velká množství výpočtů se hodí použít rychlý programovací jazyk (zpravidla kompilovaný do strojového kódu).

Ale také to byla předpokládaná jednoduchosti a rychlost implementace z pohledu autora práce. Tzn. bylo vhodné používat prostředky, se kterými se již setkal, a nemusel se tak kromě samotné implementace ještě seznamovat s novým prostředím.

\subsection{Programovací jazyk}

Pro vývoj aplikace byl zvolen programovací jazyk \emph{Java}\src. Jednak autor s tímto programovacím jazykem už v minulosti pracoval.

Další důvodem je i to, že standardní \emph{SDK} (zkratka \emph{software development kit}, volně česky \emph{standardní knihovna}) tohoto jazyka obsahuje poměrně dobře popsaně \emph{API} (zkratka \emph{application programming interface}, volně česky \emph{rozhraní pro programování aplikací}), které programům umožňuje jednoduché vytváření \emph{GUI} (zkratka \emph{graphical user interface}, česky \emph{grafické uživatelské rozhraní}). To zahrnuje zejména zobrazování počítačových oken a různé způsoby jak obsah těchto oken definovat a vykreslovat. Zmíněné \emph{API} se v \emph{Javě} nazývá \emph{Java Swing}.\src

Pro produkční software by byla samozřejmě důležitá především rychlost simulace. Z tohoto hlediska by bylo vhodnější použít např. programovací jazyky \emph{C++} nebo \emph{Rust}, které jsou na rozdíl od \emph{Javy} kompilované přímo do strojového kódu, a proto jsou schopné jednoduché aritmetické výpočty provádět podstatně rychleji.

\subsection{Vývojové prostředí}

Vývoj aplikace probíhal v prostředí \emph{IntelliJ IDEA} od české firmy \emph{JetBrains}. Pro kompilaci a správu závislostí (anglicky \emph{dependency management}, tj. hlavně externí knihovny) byl použit nástroj \emph{Maven}. Důvodem byla čistě jen autova znalost těchtowa software. 
