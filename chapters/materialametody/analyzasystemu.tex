\section{Analýza systému}

\subsection{Vztažná soustava}

Nejprve je nutné si zvolit vztažnou soustavu pro analýzu systému. Použijeme \emph{kartézskou soustavu souřadnic} s počátkem $O$ a osami $x$ a $y$ (soustava $Oxy$). Optický systém do ní umístíme tak, aby se většina paprsků šířila ve směru $+x$. Poté lze (téměř) libovolný paprsek $p$ \emph{jednoznačně} popsat následující rovnicí:

\[ p:\ \ y = k \cdot x + q \]

Přičemž $x$ a $y$ jsou souřadnice bodů, jimiž paprsek prochází, $q$ je $y$-souřadnice průsečíku paprsku se zvolenou osou $y$ a $k$ je směrnice paprsku. Ta se dá definovat i v závislosti na úhlu $\alpha$ mezi paprskem a osou $x$:

\[ k = tan\ \alpha \]

Problémy s nejednoznačností nastávají, když se některé paprsky šíří ve směru $-x$. Zároveň paprsky rovnoběžné s osou $y$ nelze pomocí výše uvedené rovnice definovat. Tyto případy však zpravidla nehrají velkou roli v nezobrazovacích optických systémech, proto je z analýzy vypouštíme.


\subsection{Definice prostorovo-fázového diagramu}

\emph{Prostorovo-fázový diagram} (\emph{pf-diagram}) zkoumá chování paprsků v blízkém okolí osy $y$. Tento diagram je tvořen body, které reprezentují jednotlivé paprsky.

Souřadnice $y$ těchto bodů je rovna hodnotě $q$ z předchozí definice, neboli souřadnici průsečíku paprsku s osou $y$. Z toho vyplývá, že jednotky (a pro uživatelskou přívětivost často i měřítko) osy $y$ pf-diagramu budou shodné jako u osy $y$ systému $Oxy$.

Naproti tomu, souřadnice $x$ vyjadřuje sklon paprsku, ať už ve formě úhlu $\alpha$, směrnice $k$, nebo sinu úhlu $sin\ \alpha$ apod.

V této části práce budeme využívat diagramy, které na ose $x$ zobrazují směrnici $k$. Jinými slovy, pro $n$-tý paprsek $p_n$, definovaný jako:

\[ p_n:\ \ y = k_n \cdot x + q_n \]

zakreslíme do pf-diagramu bod $X_n$: 

\[ X_n\ [\ k_n,\ q_n\ ] \]

Příklad takového pf-diagramu je znázorněn na následujících obrázcích:

\doubleimage{
    \image[scale=0.15]{analyzasystemu_p1_scena.png}{Scéna s několika barevně označenými paprsky}
}{
    \image[scale=0.15]{analyzasystemu_p1_etendue.png}{Vizualizace paprsků z předchozího obrázku v pf-diagramu}
}

Všimněme si, že modrý paprsek je rovnoběžný s paprskem fialovým. To má za následek shodnost $x$-souřadnic bodů, jež tyto paprsky reprezentují v pf-diagramu. Podobně, průsečík červeného paprsku s osou $y$ je totožný s průsečíkem modrého paprsku s osou $y$. To způsobuje, že odpovídající body mají stejnou $y$-souřadnici.

Pro lepší pochopení optických systémů se hodí možnost odpoutat pf-diagram od osy $y$ a vypočítávat jej pro libovolnou s ní rovnoběžnou přímku. Takovouto \emph{detekční přímku} $d$ lze definovat:

\[ d:\ \ x = c \]

kde $c$ je konstanta, o kterou je přímka posunutá od osy $y$. Bod tohoto pf-diagramu pak má souřadnice:

\[ X_n\ [\ k_n,\ p_n(\ c\ )\ ] \]

Příkladem bude tatáž scéna jako výše, ovšem s detekční přímkou pro $x = 1\ \mathrm{m}$:

\doubleimage{
    \image[scale=0.15]{analyzasystemu_p2_scena.png}{Scéna s několika barevně označenými paprsky a detekční přímkou (zelená) pro $x = 1\ \mathrm{m}$}
}{
    \image[scale=0.15]{analyzasystemu_p2_etendue.png}{Vizualizace paprsků z předchozího obrázku v pf-diagramu pro přímku $x = 1\ \mathrm{m}$}
}

V diagramu samozřejmě zobrazujeme pouze ty sekce paprsků, které opravu protínají detekční přímku.

\subsection{Vlastnosti prostorovo-fázového diagramu}



\doubleimage{
    \image[scale=0.15]{analyzasystemu_p3_scena.png}{Vějířovitý zdroj světla s několika znázorněnými paprsky (červená) a tři vyznačené detekční přímky, pro $x = 0\ \mathrm{m}$, $x = 2\ \mathrm{m}$ a $x = 4\ \mathrm{m}$ (zelená)}
}{
    \image[scale=0.15]{analyzasystemu_p3_etendue1.png}{Vizualizace pf-diagramu pro levou detekční přímku $x = 0\ \mathrm{m}$ v předchozím obrázku}
}

\doubleimage{
    \image[scale=0.15]{analyzasystemu_p3_etendue2.png}{Vizualizace pf-diagramu pro prostřední detekční přímku $x = 2\ \mathrm{m}$ v předchozím obrázku}
}{
    \image[scale=0.15]{analyzasystemu_p3_etendue3.png}{Vizualizace pf-diagramu pro pravou detekční přímku $x = 4\ \mathrm{m}$ v předchozím obrázku}
}



\singleimage{
    \image[scale=0.15]{analyzasystemu_p4_scena.png}{Zdroj rozvnoběžného svazku paprsků, z nichž některé jsou znázorněny (červená) a dvě vyznačené detekční přímky, pro $x = 0\ \mathrm{m}$ a $x = 4\ \mathrm{m}$ (zelená)}
}

\doubleimage{
    \image[scale=0.15]{analyzasystemu_p4_etendue1.png}{Vizualizace pf-diagramu pro levou detekční přímku $x = 0\ \mathrm{m}$ v předchozím obrázku}
}{
    \image[scale=0.15]{analyzasystemu_p4_etendue2.png}{Vizualizace pf-diagramu pro pravou detekční přímku $x = 4\ \mathrm{m}$ v předchozím obrázku}
}



\singleimage{
    \image[scale=0.15]{analyzasystemu_p5_scena.png}{Úsečkový zdroj světla s několika znázorněnými paprsky (červená) a dvě vyznačené detekční přímky, pro $x = 0\ \mathrm{m}$ a $x = 4\ \mathrm{m}$ (zelená)}
}

\doubleimage{
    \image[scale=0.15]{analyzasystemu_p5_etendue1.png}{Vizualizace pf-diagramu pro levou detekční přímku $x = 0\ \mathrm{m}$ v předchozím obrázku}
}{
    \image[scale=0.15]{analyzasystemu_p5_etendue2.png}{Vizualizace pf-diagramu pro pravou detekční přímku $x = 4\ \mathrm{m}$ v předchozím obrázku}
}



\doubleimage{
    \image[scale=0.15]{analyzasystemu_p6_scena.png}{Úsečkový zdroj světla s několika znázorněnými paprsky (červená), clona (šedá) a vyznačená detekční přímka $x = 4\ \mathrm{m}$ (zelená)}
}{
    \image[scale=0.15]{analyzasystemu_p6_etendue.png}{Vizualizace pf-diagramu scény na předchozím obrázku}
}



\doubleimage{
    \image[scale=0.15]{analyzasystemu_p7_scena.png}{Úsečkový zdroj světla s několika znázorněnými paprsky (červená), zrcadlo (modrá) a vyznačená detekční přímka $x = 4\ \mathrm{m}$ (zelená)}
}{
    \image[scale=0.15]{analyzasystemu_p7_etendue.png}{Vizualizace pf-diagramu scény na předchozím obrázku, obsahuje neodražené paprsky (červená) a odražené paprsky (modrá)}
}


\subsection{Étendue}

Z důvodu přehlednosti budeme osu $x$ popisovat pomocí stupňů $^{\circ}$. Ovšem, abychom dosáhli zobrazení sin $\alpha$, použité měřítko nebude lineární.

