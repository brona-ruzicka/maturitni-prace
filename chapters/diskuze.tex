\chapter{Diskuze}

Celkově můžeme pozorovat, že vytvořená aplikace má několik nedostatků, které by se daly rozdělit do dvou hlavních kategorii. Vzhledem k cíli této práce, kterýmžto je pouze ověřit použitelnost konceptu, jim však nepřikládáme velkou váhu.

\section{Nedostatky návrhu aplikace}

Tyto problémy se týkají samotného konceptu aplikace.

Prvním mírným nedostatkem je samozřejmě již několikrát zmiňovaná absence pokročilejších funkcí, tj. složitější geometrie, další typy interakcí apod. (viz kapitola \fullref{sec:vymezeniproblemu}). Ovšem aplikace je navržena tak, že uživatel se základní znalostí programování je schopen bez většího zásahu potřebné možnosti doplnit.

Dalším problémem je nutnost definice scény přímo v kódu a v nikoli nějakém externím souboru (podkapitola \fullref{sub:architekturaaplikace_zadanisceny}), jako tomu často bývá u podobných simulačních aplikací. To má za následek nutnost kompilovat program při každém spuštění a vyžaduje po uživateli minimálně základní znalost programování. V ideálním případě by náš program měl podporovat alespoň některé hojně používané formáty v oblati simulace optiky.

V souvislosti s tím bylo též vhodné umožnit uživateli kromě změny pozice detekční přímky i úpravu scény přímo v běžící aplikaci, aby se opět nemusel celý program pouštět znovu.

\section{Nedostatky implementace aplikace}

Tato kategorie se zabývá výpočetními chybami, které se týkají až hotové aplikace. Mohou být způsobeny buďto limitacemi přímo programovacího jazyka \emph{Java} nebo neoptimálním řešením samotného autora aplikace.

První z chyb je dobře viditelná na některých vizualizacích pf-diagramu. Specificky to jsou obr. \ref{fig:vysledky_1a_etendue.png} a \ref{fig:vysledky_1b_etendue.png}. Nad i pod hlavní částí obrazce jsou viditelné nezřetelné čáry. Všimněme si, že podle zadání obou příkladů v související kapitole \fullref{sec:svetlovody} se celý zdroj světla nachází uvnitř reflektoru, ještě s mírnou rezervou. Ovšem paprsky, které jsou zmíněnými čárami znázorněny se zcela jistě nacházejí mimo světlovod. Jediné vysvětlení je, že při sledování chodu tyto paprsky nějakým způsobem prošly skrze zrcadlo, aniž by se odrazily.

To je způsobeno kombinací několika faktorů. Tím prvním je čistě způsob implementace ray tracing procesu (podkapitola \fullref{sub:architekturaaplikace_simulacepaprsku}). Při něm totiž po nalezení dalšího průsečíku program kontroluje, jestli není příliš blízko poslednímu známému bodu. To má za cíl zabránit dvojitému odrazu (tzn. paprsek se odrazí dvakrát od toho stejného zrcadla na stejném místě a vypadá to, jako kdyby jen prošel skrz). Dalo by se říci, že použitý způsob jak zabránit chybám v simulaci sám vytvoří další chyby. Ovšem tento nový druh chyb je podstatně méně četný a vyskytuje se jen ve specifických případech. Druhým důvodem jsou nepřesnosti ve výpočtech s tzv. floating-point číselnými formáty. \parencite{ieee2008floating} Kvůli nim nemůžeme v při sledování chodu paprsků jednoduše porovnat jestli poslední známý bod je shodný s právě nalezeným průsečíkem. To protože v mnoha případech není, i když čísla se liší jen o velmi malé zlomky. Proto se musíme uchýlit ke kontrolování vzdáleností těchto dvou bodů.

Druhá chyba implementace programu se tentokrát týká výpočtu étendue. Ze \emph{zákona zachování étendue} (kapitola \fullref{sec:analyzasystemu}) vyplývá, že pro jeden a ten samý zdroj, nehledě na okolní optické prvky a volbu detekční přímky, bude étendue konstantní. Ovšem, jak je vidět na příkladech se světlovodem (obr. \ref{fig:vysledky_1a_etendue.png} a \ref{fig:vysledky_1b_etendue.png}), ale i s parabolickým reflektorem (obr. \ref{fig:vysledky_2a_etendue.png}, \ref{fig:vysledky_2b_etendue.png} a \ref{fig:vysledky_2c_etendue.png}), námi vypočtené étendue tomu ne vždy odpovídá.

Nepřesnost těchto hodnot je způsobena zvoleným způsobem výpočtu. Étendue počítáme jako plochu pf-diagramu a ten zobrazujeme po ve zvolené mřížce. Pro lepší kvalitu výpočtu by bylo vhodné snížit velikost jednotlivých políček mřížky. Ovšem to by vyžadovalo zvýšení počtu simulovaných paprsků, aby se zamezilo případům, že nějaké políčko zcela jasně uvnitř obrazce nebude zasaženo žádným paprskem. Simulace vyššího množství paprsků by zase trvala déle, proto v tomto případě bylo nutné najít kompromis mezi rychlostí a přesností výpočtu.


\section{Celková užitečnost}

I přes zmíněné nedostatky se aplikace prokázala jako užitečná. Jako příklad nám poslouží simulace z kapitoly \fullref{sec:parabolickereflektory}. Předpokládejme, že naším cílem je soustředit co nejvíce světla ve směru osy paraboly. Pokud porovnáme pf-diagramy pro různé pozice ohnisek, je zřejmé, že sám o sobě je pro nás nejlepší reflektor se zdrojem v ohnisku (obr. \ref{fig:vysledky_2a_scena.png}). A to protože centra odražených paprsků jsou přímo uprostřed pf-diagramu, tedy mají sklon blízký nule.

Ovšem pokud bychom měli možnost napravo od reflektoru umístit ještě zobecněnou (\emph{freeformovou}) čočku, z pf-diagramu zjistíme, že nejlepších výsledků dosáhneme pokud bude ohnisko posunuté doprava, neboli dovnitř reflektoru (obr. \ref{fig:vysledky_2b_scena.png}). A to z toho důvodu, že útvary odražených paprsků se posunou na stranu směrem ke zbytku neodražených paprsků. Takovýto tvar může zobecněná čočka posunout zpět směrem k nulovému sklonu. Nejenže vrátí centra odražených paprsků zpět na nulový sklon (jak tomu bylo u předchozího příkladu), ale společně s nimi zlomí i všechny neodražené paprsky blíže k ose reflektoru. Tudíž ve výsledku celý systém soustředí více světla ve směru osy $x$. Takováto konfigurace by tedy stála za další prozkoumání.

Na užitečnosti vytvořené aplikace se shodlo i několik odborníků zabývajících se návrhem nezobrazovací optiky. (P. Lobaz, osobní komunikace, 24. 2. 2023)
